\documentclass[11pt]{ctexart}
\usepackage{tabularx}
\usepackage{array}
\usepackage{bm}
\usepackage{hyperref}
\usepackage{amsmath}
\usepackage{algorithm}
%\usepackage{algorithm2e}
\usepackage{algorithmicx}
\usepackage{algpseudocode}
\usepackage{fancyhdr}
\pagestyle{fancy}

\hypersetup{hypertex=true,
	colorlinks=true,
	linkcolor=blue,
	anchorcolor=blue,
	citecolor=blue}
\fancyhf{}
\chead{\textbf{算法设计与分析}}
\fancyhead[r]{\bfseries\thepage}
\fancyhead[l]{\bfseries\rightmark}
\renewcommand{\headrulewidth}{0.4pt} % 注意不用 \setlength
\renewcommand{\footrulewidth}{0pt}

\floatname{algorithm}{算法}
\renewcommand{\algorithmicrequire}{\textbf{输入:}}
\renewcommand{\algorithmicensure}{\textbf{输出:}}

\begin{titlepage}
	\title{\Huge\textbf{算法设计与分析 作业二}\\}
	\author{\Large\textbf{作者}:吴润泽 \and{\Large\textbf{学号}:181860109}\\
	\\
	\and {\Large\textbf{Email}:181860109@smail.nju.edu.cn}\\}
	\date{\Large\today}
\end{titlepage}

\begin{document}
	\maketitle

	\tableofcontents
		\newpage
	\section*{PART I}
	\markright{PART I}
	\addcontentsline{toc}{section}{PART I}
	\subsection*{problem 6.8}
	\markright{problem 6.8}
	\addcontentsline{toc}{subsection}{problem 6.8}
		
	\paragraph{算法分析}假定n总是k的倍数,且n和k都是2的幂。\\
	\hspace*{20pt}利用快排的思想,将数组从中间划分为两段$A[0\cdots n/2],\ A[n/2+1\cdots n]$,且左段元素小于右段元素。\\
	\hspace*{20pt}对于子序列继续递归划分,得到$A[0\cdots n/2^m],A[n/2^m+1\cdots n/2^{m-1}]\cdots A[n-n/2^m+1\cdots n]$,当$2^m=k\rightarrow m=\log k$时,划分完成。\\
	\hspace*{20pt}因此寻找中位数划分的函数时间复杂度应为$O(n)$,划分函数的递归方程为$W(n)=2W(n/2)+O(n)$,划分左右子段$\log k$次,方能使得总的时间复杂度达到$O(n\log k)$。
	\paragraph{具体算法实现请见下页}算法k-sorted:算法~\ref{k-sorted}
	\paragraph{算法时间复杂度}
	对于findk\_pos,每次递归代价为$O(n)$,每次子问题缩小为原来一半的规模,且子问题只有一个,可列出递归方程为$T(n)=T(n/2)+O(n)$,由主定理可以得出$T(n)=O(n)$。\\
	\hspace*{20pt}对于k\_sorted,每次递归代价为$O(n)$,每次子问题缩小为原来一半,而需要划分左右两序列,子问题为两个,可列出递归方程为$W(n)=2W(n/2)+O(n)$,注意结束条件为递归调用了$\log k$层,每层代价均为$O(n)$,因此时间复杂度为$O(n\log k)$满足题目要求。
	
	\begin{algorithm}
		\caption{k-sorted算法}
		\label{k-sorted}
		\begin{algorithmic}[1]
			\Require 待划分序列$A[1\cdots n]$,划分段数$k$
			\Ensure 划分后的的序列$A$
			\Function{findk\_pos}{$A,k\_pos,begin,end$} \verb|\\|返回该段数组第k小\\
			\verb|/*|
			利用快排思想,选定一个key,将大于key的元素放在其右边,小于key放于左边。\\
			判断key插入的位置是否为k,如果是则函数返回,如果插入位置大于k说明第k小位于左子序列对左边递归寻找,否则对右子序列递归寻找。
			\verb|*/|
			\State $split \gets begin,key \gets A[begin]$
			\For{$i\gets begin+1\ to\ end$}
				\State {$A[i]\leq key\ ?\ swap(A[++split],A[i])$}
			\EndFor
			\State{$split>k\_pos\ ?\ \Return\ findk\_pos(A,k\_pos,begin,split-1)$}
			\State{$split<k\_pos\ ?\ \Return\ findk\_pos(A,k\_pos,split+1,end)$}
			\State \Return{$split$}
			\EndFunction
			\Function {K\_SORTED}{$A,begin,end,k,count=1$}\\
			\verb|/*|
			count记录当前的段数,每次调用findk\_pos,A被分为[begin,mid]和[mid+1,end]两段,段数变为原来两倍,且左段元素小于右段,调用层数达到logk层算法结束,否则继续划分左右子序列
			\verb|*/|
			\State{$mid\gets (end-begin)/2+begin,count\gets count*2$}
			\State{$findk\_pos(A,mid,begin,end)$}
			\If{$count==k$}\\划分k段,算法结束
			\State \Return{$A$}
			\EndIf
			\State{$k\_sorted(A,begin,mid,k,count)$}
			\State{$k\_sorted(A,mid+1,end,k,count)$}
			\EndFunction
		\end{algorithmic}
	\end{algorithm}

	\newpage
	\subsection*{problem 6.9}
	\markright{problem 6.9}
	\addcontentsline{toc}{subsection}{problem 6.9}
	\paragraph{算法分析}同样利用快速排序的思想,令螺钉为A,螺母为B:\\
	\hspace*{20pt}1. 在A数组中拿一个,根据A和螺母的大小关系,可以分成三部分,B1:比螺钉小的,B2:比螺钉大的,B3:完全匹配的。\\
	\hspace*{20pt}2. 用B3,同样可以把A分为三部分,A1:比螺母小的,A2:比螺母大的,A3:完全匹配的。\\
	\hspace*{20pt}3. B1与A1匹配,B2与A2匹配,分别执行上述算法,直至全部匹配。
	\paragraph{具体算法实现请见下页}算法match:算法~\ref{match算法}
	\paragraph{算法时间复杂度}对于每次递归代价:\\
	\hspace*{20pt}首先寻找分割点,遍历了一次数组,时间复杂度为$O(n)$。\\
	\hspace*{20pt}之后根据分割点遍历A,B两数组将其分为两部分,时间复杂度为$O(n)$。\\
	\hspace*{20pt}最后将分割后两序列进行递归操作继续划分。因此每次递归操作总的代价为$O(n)$。\\
	\hspace*{20pt}因此可推得算法的递推方程为$T(n)=2T(n/2)+O(n)$。根据主定理可得时间复杂度为$O(n\log n)$满足题目要求。
	\begin{algorithm}
		\caption{match算法}
		\label{match算法}
		\begin{algorithmic}[1]
			\Require 螺钉数组$A$,螺母数组$B$
			\Ensure 螺钉螺母对应下标一一匹配后的数组
			\Function{MATCH}{$A,B,l,r$}\\
			\verb|/*|找到分割点,mark记录B等于A首元素的下标,\\
			count记录B中小于A的个数\verb|*/|
			\State {$count \gets 0, mark\gets 0$}
			\For{$i\gets l \ to\ r$}
			\State{$A[l]==B[i]\ ?\ mark=i$}
			\State{$A[l]>B[i]\ ?\ count+=1$}
			\EndFor\\
			\verb|/*|为B和A的左半部分分配count个元素\verb|*/|
			\State{$swap(A[l],A[l+count])\ swap(B[mark],B[l+count])$} 
			\State{$mark\gets mark+count, i\gets l,\ j \gets r$}
			\While{$i<mark$\ \textbf{and}\ $j>mark$}\verb|\\|将a分成两部分
			\While{$i<mark$\ \textbf{and}\ $a[i]<b[mark]$}\State{$i\gets i+1$}
			\EndWhile
			\While{$j>mark$\ \textbf{and}\ $a[j]<b[mark]$}\State{$j\gets j-1$}
			\EndWhile
			\State{$swap(a[i++],a[j--])$}
			\EndWhile
			
			\State{$i\gets l,\ j \gets r$}
			\While{$i<mark$\ \textbf{and}\ $j>mark$}\verb|\\|将b分成两部分
			\While{$i<mark$\ \textbf{and}\ $b[i]<a[mark]$}\State{$i\gets i+1$}
			\EndWhile
			\While{$j>mark$\ \textbf{and}\ $b[j]<a[mark]$}\State{$j\gets j-1$}
			\EndWhile
			\State{$swap(b[i++],b[j--])$}
			\EndWhile
			\State{$l<mark\ ? \ match(A,B,l,mark-1)$}
			\State{$r<mark\ ? \ match(A,B,mark+1,r)$}
			\EndFunction
		\end{algorithmic}
	\end{algorithm}
	\newpage
	\subsection*{problem 6.10}
	
	\markright{problem 6.10}
	\addcontentsline{toc}{subsection}{problem 6.10}
	\subsection*{problem 6.13}
	\markright{problem 6.13}
	\addcontentsline{toc}{subsection}{problem 6.13}
	\subsection*{problem 6.15}
	\markright{problem 6.15}
	\addcontentsline{toc}{subsection}{problem 6.15}
	\newpage
	\section*{PART II}
	\markright{PART II}
	\addcontentsline{toc}{section}{PART II}
	\subsection*{problem 7.1}
	\markright{problem 7.1}
	\addcontentsline{toc}{subsection}{problem 7.1}
	\newpage
	\subsection*{problem 7.2}
	\markright{problem 7.2}
	\addcontentsline{toc}{subsection}{problem 7.2}
	\subsection*{problem 7.3}
	\markright{problem 7.3}
	\addcontentsline{toc}{subsection}{problem 7.3}
	\subsection*{problem 7.4}
	\markright{problem 7.4}
	\addcontentsline{toc}{subsection}{problem 7.4}
	\subsection*{problem 7.6}	
	\markright{problem 7.6}
	\addcontentsline{toc}{subsection}{problem 7.6}
\end{document}