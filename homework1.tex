\documentclass[11pt]{ctexart}
\usepackage{tabularx}
\usepackage{array}
\usepackage{bm}
\usepackage{hyperref}
\usepackage{amsmath}
\usepackage{algorithm}
%\usepackage{algorithm2e}
\usepackage{algorithmicx}
\usepackage{algpseudocode}
\usepackage{fancyhdr}
\pagestyle{fancy}



\fancyhf{}
\chead{\textbf{算法设计与分析}}
\fancyhead[r]{\bfseries\thepage}
\fancyhead[l]{\bfseries\rightmark}
\renewcommand{\headrulewidth}{0.4pt} % 注意不用 \setlength
\renewcommand{\footrulewidth}{0pt}

\floatname{algorithm}{算法}
\renewcommand{\algorithmicrequire}{\textbf{输入:}}
\renewcommand{\algorithmicensure}{\textbf{输出:}}

\begin{titlepage}
	\title{\Huge\textbf{算法设计与分析 作业一}\\}
	\author{\Large\textbf{作者}:吴润泽 \and{\Large\textbf{学号}:181860109}\\
	\\
	\and {\Large\textbf{Email}:181860109@smail.nju.edu.cn}\\}
	\date{\Large\today}
\end{titlepage}

\begin{document}
	\maketitle
	\newpage
	\tableofcontents
	
	\section*{PART I}
	\markright{PART I}
	\addcontentsline{toc}{section}{PART I}
	\subsection*{problem 1.1}
	\markright{problem 1.1}
	\addcontentsline{toc}{subsection}{problem 1.1}
	\subsubsection*{(1)}
	\begin{algorithm}
		\caption{对三个数进行排序}
		\label{Sort}
		\begin{algorithmic}[1]
			\Require 含有三个各不相同的整数的序列$A=\{a,b,c\}$
			\Ensure 从小到大排好序的序列$A$
			\Function {ThreeNumSort}{$A$}
			\For{$i\gets 2 \ to \ 3 $}
			\State $temp \gets A[i]$
			\State $j \gets i-1$
			\While{$A[j]>temp$ \textbf{and} $j>0$}
			\State $A[j+1] \gets A[j]$
			\State $j \gets j-1$
			\EndWhile
			\State $A[j+1] \gets temp$
			\EndFor
			\EndFunction
		\end{algorithmic}
	\end{algorithm}
	
	\subsubsection*{(2)}	
	\paragraph{最坏情况比较次数为:}3次,
	\textbf{平均情况比较次数为}:{\Large$ \frac{2+2+3+3+3+3}{6}=\frac{8}{3}$}次。
	
	\subsubsection*{(3)}
	\paragraph{在最坏情况下:}至少需要比较3次,最坏情况下,即每次进入 while 循环当前i指针位置都要与其前面元素进行比较。
	假设三个数按照大小依次为$a,b,c$,
	则初始未排序序列共有6种即$\{abc\},\{acb\},\{bac\},\{bca\},\{cab\},\{cba\}$。\\
	其中$\{acb\},\{bca\},\{cab\},\{cba\}$四种序列,根据算法\ref{Sort},在for循环体中\\
	,第一次循环需要比较 1 次,第二次循环需要比较 2 次,因此最坏情况下需要比较 3 次
	\subsection*{problem 1.2}
	\markright{problem 1.2}
	\addcontentsline{toc}{subsection}{problem 1.2}
	\subsubsection*{(1)}
	\begin{algorithm}
		\caption{三个数的中位数}
		\label{middle}
		\begin{algorithmic}[1]
			\Require 含有三个各不相同的整数$\{a,b,c\}$
			\Ensure 返回序列的中位数
			\Function {ThreeNumMiddle}{$a,b,c$}
			\If{$a<b$}
			\State $swap(a,b)$
			\EndIf
			\If{$b>c$}
			\State \Return{$b$}
			\ElsIf{$a<c$}
			\State \Return{$a$}
			\Else 
			\State \Return{$c$}
			\EndIf
			\EndFunction
		\end{algorithmic}
	\end{algorithm}
	
	\subsubsection*{(2)}	
	\paragraph{在算法 \ref{middle} 中}
	最坏情况下比较次数为:3次 \\
	平均情况比较次数为:{\Large$ \frac{2+2+3+3+3+3}{6}=\frac{8}{3}$}
	
	\subsubsection*{(3)}
	\paragraph{在最坏情况下:} 首先第一个 IF 语句体中,确定前两个数字的大小关系,如果之后不满足第二个数大于第三个数,则此时中位数在第一个数和第三个数之间,仍需一次比较判断,故最坏情况下需要比较三次。
	\subsection*{problem 1.3}
	\markright{problem 1.3}
	\addcontentsline{toc}{subsection}{problem 1.3}
	\subsubsection*{(1)}
	\paragraph{反例:}假设全集 $U={1,2,3,4}$,U 的子集组成的集合族	$S={S_{1},S_{2},S_{3},S_{4}}$,其中$S_{1}=\{1,2,3\},S_{2}=\{1,2,4\},S_{3}=\{1,2,5\},S_{4}=\{4,5\}.$\\
	根据题中所给算法得到的覆盖为$\{S_{1},S_{2},S_{3}\}$,而正确的最小覆盖应为$\{S_{1},S_{4}\}.$
	\subsubsection*{(2)}	
	\begin{algorithm}
		\caption{最小集合覆盖:贪心近似解}
		\label{cover}
		\begin{algorithmic}[1]
			\Require 待覆盖的集合U,U的子集组成的一个集合族S
			\Ensure 作为覆盖集的结果C
			\Function {Cover}{$U,S$}
			\State $members \gets U$
			\State $subsets \gets S$
			\State $covering \gets \{\}$
			\While{$size(members)>0 $ \textbf{and} $size(subsets)>0$}
			\State $intersection \gets set\_intersection(members,subsets)$\footnotemark
			\State $members \gets remove(members,intersection)$
			\State $subsets \gets  remove(subsets,intersection)$\footnotemark
			\State $covering \gets add(covering,intersection)$\footnotemark
			\EndWhile
			\If{$size(members)>0$}
			\State \Return{$-1$}\footnotemark
			\EndIf
			\State \Return{$covering$}
			\EndFunction
		\end{algorithmic}
	\end{algorithm}
	\footnotetext[1]{在subsets中找出能够覆盖到members的最大交集}
	\footnotetext[2]{在members中将intersection覆盖元素去掉,在subsets中将intersection集合去掉}
	\footnotetext[3]{在covering中加入intersection作为覆盖集成员}
	\footnotetext[4]{如果members中仍然存在未被覆盖的元素,那么不可能实现完全覆盖}
	
	\subsubsection*{(3)}
	不能总是得到最小覆盖。
	\paragraph{反例:}假设全集 $U={1,2,3,4,5,6,7}$,U 的子集组成的集合族	$S={S_{1},S_{2},S_{3},S_{4}}$,其中$S_{1}=\{1,2,3,4\},S_{2}=\{1,2,5\},S_{3}=\{3,4,6\},S_{4}=\{7\}.$\\
	根据上述算法 \ref{cover} 得到的覆盖为$\{S_{1},S_{2},S_{3},S_{4}\}$,而正确的最小覆盖应为$\{S_{2},S_{3},S_{4}\}.$因此并不能总是得到正确解。
	\subsection*{problem 1.5}
	\markright{problem 1.5}
	\addcontentsline{toc}{subsection}{problem 1.5}
	\paragraph{Induction on n:}\label{Induction_one}
	\begin{itemize}
		\item \textbf{Base case}
		\begin{itemize}
			\item {$n=0, \ for\ any$\\
				$x:\ HORNER(A[0],x)=a_0$}
		\end{itemize}
		\item \textbf{Assumption}
		\begin{itemize}
			\item {$n=k,\ for\ any\ x:$\\
				$ HORNER(A[0...k],x)=a_k x^k+ a_{k-1}x^{k-1}+...+a_1x+a_0$}
		\end{itemize}
		\item \textbf{Induction}
		\begin{itemize}
			\item {
				$n=k+1,\ for\ any\ x\ Assume\ B[0...k]\gets A[1...k+1]: $
			}
			\item[] {
				$
				\begin{aligned}
				& HORNER(A[0...k+1],x)\\
				& =x*HORNER(B[0...k],x)+a_0\\
				& =x*(a_{k+1} x^k+ a_{k}x^{k-1}+...+a_2x+a_1)+a_0\\
				& =a_{k+1}x^{k+1}+a_{k}x^{k}+...+a_2x^2+a_1x+a_0
				\end{aligned}
				$
			}
		\end{itemize}
	\end{itemize}
\newpage
	\subsection*{problem 1.6}
	\markright{problem 1.6}
	\addcontentsline{toc}{subsection}{problem 1.6}
	\subsubsection*{(1)}
	\paragraph{Induction on z:}\label{Induction_two}
	\begin{itemize}
		\item \textbf{Base case}
		\begin{itemize}
			\item {$z=0, \ for\ any\ y\geq0:\ INT\_MULT(y,0)=0$}
			\item{$z=1, \ for\ any\ y\geq0:$}
			\item[]{
				$
				\begin{aligned}
				&INT\_MULT(y,1)\\
				&=INT\_MULT(2y,\lfloor\frac{1}{2}\rfloor) + y\cdot(1\mod2)\\
				&=INT\_MULT(2y,0)+y\\
				&=y
				\end{aligned}
				$
			}
			
		\end{itemize}
		\item \textbf{Assumption}
		\begin{itemize}
			\item {$for\ any\ z \geq k,\ for\ any\ y \geq0:$}
			\item[] {
				$
				\begin{aligned}
				&INT\_MULT(y,z)\\
				&=INT\_MULT(2y,\lfloor\frac{z}{2}\rfloor) + y\cdot(z\mod2)\\
				&=y \cdot z
				\end{aligned}
				$			
			}
			
		\end{itemize}
		\item \textbf{Induction}
		\begin{itemize}
			\item {$z=k+1,\ for\ any\ y:$}
			\item[]{
				$
				\begin{aligned}
				& INT\_MULT(y,k+1)\\
				& =INT\_MULT(2y,\lfloor\frac{k+1}{2}\rfloor) + y\cdot(k+1\mod2)\\
				& =y\cdot(2 \lfloor\frac{k+1}{2}\rfloor+(k+1)\mod2)\\
				& =y\cdot (k+1)
				\end{aligned}
				$
			}
		\end{itemize}
	\end{itemize}
	\subsubsection*{(2)}
	\paragraph{Induction on z:}
	与证明\ref{Induction_two}方法类似,将其中的c=2替换为c即可。
	
	\subsection*{problem 1.7}
	\markright{problem 1.7}
	\addcontentsline{toc}{subsection}{problem 1.7}
	\paragraph{平均时间复杂度:}
	\begin{equation}
	A(n)=\sum_{I\in D(n)} Pr(I)f(I)=\frac{1}{n}\cdot10\cdot\frac{n}{4}+\frac{2}{n}\cdot20\cdot\frac{n}{4}+\frac{1}{2n}\cdot30\cdot\frac{n}{4}+\frac{1}{2n}\cdot n\cdot\frac{n}{4}=\frac{65}{4}+\frac{n}{8} \notag
	\end{equation}
	\newpage
	\section*{PART II}
	\markright{PART II}
	\addcontentsline{toc}{section}{PART II}
	\subsection*{problem 2.2}
	\markright{problem 2.2}
	\addcontentsline{toc}{subsection}{problem 2.2}
	\paragraph{证明:}
	\begin{equation*}
	\begin{aligned}
	&n \geq 1,let\ 2^k \leq n \leq 2^{k+1}-1,\\
	&\Rightarrow	k<\log(2^k+1)\leq\log(n+1)\leq\log(2^{k+1})=k+1&\ (1)\\
	&(1)\rightarrow \lceil\log(n+1)\rceil=k+1\\
	&\Rightarrow k+1=\log(2^k)+1\leq\log(n)+1\leq\log(2^(k+1)-1)+1<k+2&\ (2)\\
	&(2)\rightarrow \lfloor\log(n)+1\rfloor=k+1\\
	\end{aligned}
	\end{equation*}
	因此,$\lceil\log(n+1)\rceil=\lfloor\log(n)+1\rfloor$,证毕。
	\subsection*{problem 2.3}
	\markright{problem 2.3}
	\addcontentsline{toc}{subsection}{problem 2.3}
	\subsubsection*{(1)}
	\paragraph{证明:}
	对于斐波那契数列:$F_n=F_{n-1}+F_{n-2}$,对于$F_{n}$的奇偶性与$F_{n-1}和F_{n-2}$的奇偶性有关。
	\subparagraph{奇偶性:}
	\begin{tabularx}{25em}%
		{|*{4}{>{\centering\arraybackslash}X|}}
		\hline
		\textbf{$F_n$} &\textbf{$F_{n-1}$} &\textbf{$F_{n-2}$}\\ \hline
		奇 & 奇 & 偶\\ \hline
		奇 & 偶 & 奇\\ \hline
		偶 &	奇 & 奇 \\ \hline
		偶 & 偶 & 偶\\ \hline
	\end{tabularx}
	\paragraph{由上表:}
	斐波那契数列前两项为1,均为奇数,因此永远不会出现偶数加偶数的情况,所以只能是前3种假发循环出现,因此偶数均出现在项数为3的倍数的位置。
	\newpage
	\subsubsection*{(2)}
	\paragraph{Induction on n:}\label{Induction_three}
	\begin{itemize}
		\item \textbf{Base case}
		\begin{itemize}
			\item {$n=2,\ F_{2}^2-F_{3}F_{1}=-1=(-1)^{3}$}
		\end{itemize}
		\item \textbf{Assumption}
		\begin{itemize}
			\item {$n=k,\ F_{k}^2-F_{k+1} F_{k-1}=(-1)^{k+1}$}
		\end{itemize}
		\item \textbf{Induction}
		\begin{itemize}
			\item {$n=k+1:$}
			\item[] {
				$
				\begin{aligned}
				&F_{k+1}^2-F_{k}F_{k+2}\\
				&=F_{k+1}(F_{k}+F_{k-1})-F_{k}F_{k+2}\\
				&=F_{k-1}F_{k+1}+F_{k}F_{k+1}-F_{k}F_{k+2}\\
				&=F_{k}^2+F_{k}F_{k+1}-F_{k}F_{k+2}-(-1)^{k+1}\\
				&=F_{k}(F_{k}+F_{k+1}-F_{k+2})+(-1)^{k+2}\\
				&=(-1)^{k+2}
				\end{aligned}
				$
			}
		\end{itemize}
	\end{itemize}
	\subsection*{problem 2.5}
	\markright{problem 2.5}
	\addcontentsline{toc}{subsection}{problem 2.5}
	\subsubsection*{(1)}
	\paragraph{证明:}\label{2.5.1}
	2-tree即结点的度为0或2,即仅有$n_0,n_2$假设共有n个结点,则$n_0+n_2=n$,同时n个结点则有n-1条边,n-1条边由度数为2的结点组成,即$2n_2=n-1=n_0+n_2-1\Rightarrow n_0=n_2+1$,证毕。
	\subsubsection*{(2)}
	\paragraph{成立}
	\paragraph{证明:}
	同上述证明\ref{2.5.1},假设n个结点,则$n_0+n_1+n_2=n$,n-1条边由度数为1和度数为2的结点组成,即$2n_2+n_1=n-1=n_0+n_1+n_2-1\Rightarrow n_0=n_2+1$,证毕。
	\subsection*{problem 2.6}
	\markright{problem 2.6}
	\addcontentsline{toc}{subsection}{problem 2.6}
	\subsubsection*{(1)}
	\begin{itemize}
		\item \bm{$O\ transitivity$}
		\item[] {$
			\begin{aligned}
			&f(n)=O(g(n)),g(n)=O(h(n)),\\
			&\rightarrow \lim_{n \to \infty}\frac{f(n)}{g(n)}=c<\infty,\lim_{n \to \infty}\frac{g(n)}{h(n)}=d<\infty\\
			&\rightarrow \lim_{n \to \infty}\frac{f(n)}{h(n)}=\lim_{n \to \infty}\frac{f(n)}{g(n)}\cdot \frac{g(n)}{h(n)}=c\cdot d<\infty\\
			&\rightarrow f(n)=O(h(n)),\ proved.
			\end{aligned}
			$}
		\item \bm{$\Omega\ transitivity$}
		\item[] {$
			\begin{aligned}
			&f(n)=\Omega(g(n)),g(n)=\Omega(h(n)),\\
			&\rightarrow \lim_{n \to \infty}\frac{f(n)}{g(n)}=c>0,\lim_{n \to \infty}\frac{g(n)}{h(n)}=d>0\\
			&\rightarrow \lim_{n \to \infty}\frac{f(n)}{h(n)}=\lim_{n \to \infty}\frac{f(n)}{g(n)}\cdot \frac{g(n)}{h(n)}=c\cdot d>0\\
			&\rightarrow f(n)=\Omega(h(n)),\ proved.
			\end{aligned}
			$}
		\item \bm{$\Theta\ transitivity$}
		\item[] {$
			\begin{aligned}
			&f(n)=\Theta(g(n)),g(n)=\Theta(h(n)),\\
			&\rightarrow \lim_{n \to \infty}\frac{f(n)}{g(n)}=c,\lim_{n \to \infty}\frac{g(n)}{h(n)}=d(0<c,d<\infty)\\
			&\rightarrow \lim_{n \to \infty}\frac{f(n)}{h(n)}=\lim_{n \to \infty}\frac{f(n)}{g(n)}\cdot \frac{g(n)}{h(n)}=c\cdot d(0<c\cdot d<\infty)\\
			&\rightarrow f(n)=\Theta(h(n)),\ proved.
			\end{aligned}
			$}
		\item \bm{$o\ transitivity$}
		\item[] {$
			\begin{aligned}
			&f(n)=o(g(n)),g(n)=o(h(n)),\\
			&\rightarrow \lim_{n \to \infty}\frac{f(n)}{g(n)}=0,\lim_{n \to \infty}\frac{g(n)}{h(n)}=0\\
			&\rightarrow \lim_{n \to \infty}\frac{f(n)}{h(n)}=\lim_{n \to \infty}\frac{f(n)}{g(n)}\cdot \frac{g(n)}{h(n)}=0\\
			&\rightarrow f(n)=o(h(n)),\ proved.
			\end{aligned}
			$}
		\item \bm{$\omega\ transitivity$}
		\item[] {$
			\begin{aligned}
			&f(n)=\omega(g(n)),g(n)=\omega(h(n)),\\
			&\rightarrow \lim_{n \to \infty}\frac{f(n)}{g(n)}=\infty,\lim_{n \to \infty}\frac{g(n)}{h(n)}=\infty\\
			&\rightarrow \lim_{n \to \infty}\frac{f(n)}{h(n)}=\lim_{n \to \infty}\frac{f(n)}{g(n)}\cdot \frac{g(n)}{h(n)}=\infty\\
			&\rightarrow f(n)=\omega(h(n)),\ proved.
			\end{aligned}
			$}
	\end{itemize}
	\subsubsection*{(2)}
	\paragraph{证明:}
	$\begin{aligned}\lim_{n \to \infty}\frac{f(n)}{f(n)}=1\end{aligned}$,对于$O,\Theta,\Omega$三种关系的定义均符合,因此满足自反性。
	\subsubsection*{(3)}
	\paragraph{证明:}
	\subparagraph{对称性:}
	$
	\begin{aligned}
	&f(n)=\Theta(g(n))
	\rightarrow \lim_{n \to \infty}\frac{f(n)}{g(n)}=c,(0<c,<\infty)\\
	&\rightarrow \lim_{n \to \infty}\frac{g(n)}{f(n)}=\frac{1}{c}(0<\frac{1}{c}<\infty)\\
	&\rightarrow g(n)=\Theta(f(n)),\ proved.
	\end{aligned}
	$\\
	由$(1),\ (2),\ (3)$可知,$\Theta$满足传递性、自反性、对称性,因此是等价关系。
	\subsubsection*{(4)}
	\paragraph{证明:}
	\subparagraph{充分性:}
	$
	\begin{aligned}
	&f(n)=O(g(n)),
	\rightarrow \lim_{n \to \infty}\frac{f(n)}{g(n)}=c<\infty\\
	&f(n)=\Omega(g(n)),
	\rightarrow \lim_{n \to \infty}\frac{f(n)}{g(n)}=c>0\\
	&\rightarrow  \lim_{n \to \infty}\frac{f(n)}{g(n)}=c,(0<c,<\infty)\\
	&\rightarrow f(n)=\Theta(g(n)),\ proved.
	\end{aligned}
	$\\
	\subparagraph{必要性:}
	$
	\begin{aligned}
	&f(n)=\Theta(g(n))\\
	&\rightarrow \lim_{n \to \infty}\frac{f(n)}{g(n)}=c,(0<c,<\infty)\\
	&\rightarrow f(n)=O(g(n)),\ f(n)=\Omega(g(n)),\ proved.\\
	\end{aligned}
	$\\
	\subsubsection*{(5)}
	\paragraph{a.}
	$
	\begin{aligned}
	f(n)=O(g(n))
	\Leftrightarrow \lim_{n \to \infty}\frac{f(n)}{g(n)}=c<\infty
	\Leftrightarrow \lim_{n \to \infty}\frac{g(n)}{f(n)}=\frac{1}{c}>0
	\Leftrightarrow g(n)=\Omega(f(n))
	\end{aligned}
	$
	\paragraph{b.}
	$
	\begin{aligned}
	f(n)=o(g(n))
	\Leftrightarrow \lim_{n \to \infty}\frac{f(n)}{g(n)}=0
	\Leftrightarrow \lim_{n \to \infty}\frac{g(n)}{f(n)}=\infty
	\Leftrightarrow g(n)=\omega(f(n))
	\end{aligned}
	$
	\subsubsection*{(6)}
	\paragraph{a.}
	$
	\begin{aligned}
	&assume A=o(g(n))\cap\omega(g(n))\neq\emptyset,f(n)\in A \\
	&f(n)=o(g(n)),\rightarrow \lim_{n \to \infty}\frac{f(n)}{g(n)}=0,
	\ f(n)=\omega(g(n)),\rightarrow \lim_{n \to \infty}\frac{f(n)}{g(n)}=\infty
	\end{aligned}
	$\\
	\\
	产生矛盾,即$o(g(n))\cap\omega(g(n))=\emptyset$
	\paragraph{b.}
	$
	\begin{aligned}
	&assume A=\Theta(g(n))\cap o(g(n))\neq\emptyset,f(n)\in A \\
	&f(n)=\Theta(g(n)),\rightarrow \lim_{n \to \infty}\frac{f(n)}{g(n)}=c,(0<c<\infty)
	\ f(n)=o(g(n)),\rightarrow \lim_{n \to \infty}\frac{f(n)}{g(n)}=0
	\end{aligned}
	$\\
	\\
	产生矛盾,即$\Theta(g(n))\cap o(g(n))=\emptyset$
	\paragraph{c.}
	$
	\begin{aligned}
	&assume A=\Theta(g(n))\cap \omega(g(n))\neq\emptyset,f(n)\in A \\
	&f(n)=\Theta(g(n)),\rightarrow \lim_{n \to \infty}\frac{f(n)}{g(n)}=c,(0<c<\infty)
	\ f(n)=o(g(n)),\rightarrow \lim_{n \to \infty}\frac{f(n)}{g(n)}=\infty
	\end{aligned}
	$\\
	\\
	产生矛盾,即$\Theta(g(n))\cap \omega(g(n))=\emptyset$
	\newpage
	\subsection*{problem 2.7}
	\markright{problem 2.7}
	\addcontentsline{toc}{subsection}{problem 2.7}
	\subsubsection*{(1)}
	\paragraph{排序:}
	$\log n<n<n\log n<n^2=n^2+\log n<n^3<n-n^3+7n^5<2^n$
	\subsubsection*{(2)}
	\paragraph{排序:}
	$\log\log n<\log n<(\log n)^2<\sqrt{n}<n<n\log n<n^{1+\varepsilon}<n^2=n^2+\log n<n^3<n-n^3+7n^5<2^{n-1}=2^n=e^n<n!$
	\subsection*{problem 2.15}
	\markright{problem 2.15}
	\addcontentsline{toc}{subsection}{problem 2.15}
	\subsubsection*{(1)}
	$\begin{aligned}
	&a=1,b=2,f(n)=1\\
	&n^{log_3 2}\approx n^{0.63},\ f(n)=O(n^{\log_3 2-0.5}),\varepsilon >0 \\
	&master\ case1:T(n)=\Theta (n^{\log_3 2})
	\end{aligned}
	$
	\subsubsection*{(2)}
	$\begin{aligned}
	&a=1,b=2,f(n)=c\log n\\
	&n^{log_2 1}=0,\forall \varepsilon >0,\ f(n)\neq\Omega(n^{\log_2 1+\varepsilon})\\
	&T(n)=T(\frac{n}{2})+c\log n\\
	&T(n)=T(\frac{n}{4})+c\log n+c(\log n -1)\\
	&\cdots\\
	&T(n)=T(1)+c\log n +c(\log n \log n-1-2-\cdots-k)\\
	&T(n)=\Theta((\log n)^2)
	\end{aligned}
	$
	\subsubsection*{(3)}
	$\begin{aligned}
	&a=1,b=2,f(n)=cn\\
	&n^{log_2 1}=0,\ f(n)=\Omega(n^{\log_2 1+1}),\varepsilon >0 \\
	&proving:\ af(\frac{n}{b})\leq df(n), n\geq N_0,d>0\\
	&f(\frac{n}{2})=c\frac{n}{2}\leq cdn,d\geq0.5\\
	&master\ case3:T(n)=\Theta (n)
	\end{aligned}
	$
	\subsubsection*{(4)}
	$\begin{aligned}
	&a=2,b=2,f(n)=cn\\
	&n^{log_2 2}=n,\ f(n)=\Theta(n^{\log_2 2})\\
	&master\ case2:T(n)=\Theta (n{\log n})
	\end{aligned}
	$
	\subsubsection*{(5)}
	$\begin{aligned}
	&a=2,b=2,f(n)=cn\log n\\
	&n^{log_2 2}=n,\forall \varepsilon >0,\ f(n)\neq\Omega(n^{\log_2 2+\varepsilon})\\
	&T(n)=2T(\frac{n}{2})+cn\log n\\
	&T(n)=2(2T(\frac{n}{4})+c\frac{n\log n -1}{2})+cn\log n\\
	&T(n)=4T(\frac{n}{4})+2cn\log n-c\\
	&\cdots\\
	&T(n)=cn\log n+cn(\log n-1)+cn(\log n-2)+\cdots+cn(\log n-k+1)\\
	&T(n)=n\log^2n-nk(k-1)/2\\
	&T(n)=\Theta(n(\log n)^2)
	\end{aligned}
	$
	\subsubsection*{(6)}
	$\begin{aligned}
	&a=2,b=2,f(n)=cn^2\\
	&n^{log_2 2}=n,\ f(n)=\Omega(n^{\log_2 2+1}),\varepsilon >0 \\
	&proving:\ af(\frac{n}{b})\leq df(n), n\geq N_0,d>0\\
	&2f(\frac{n}{2})=c\frac{n^2}{2}\leq cdn^2,d\geq0.5\\
	&master\ case3:T(n)=\Theta (n^2)
	\end{aligned}
	$
	\subsubsection*{(7)}
	$\begin{aligned}
	&a=49,b=25,f(n)=cn^{\frac{3}{2}}\log n\\
	&n^{log_{25}49}\approx n^{1.21},\ f(n)=\Omega(n^{log_{25}49+0.2}),\varepsilon >0 \\
	&proving:\ af(\frac{n}{b})\leq df(n), n\geq N_0,d>0\\
	&49f(\frac{n}{25})=49c(\frac{n^2}{625}(\log n-2\log 5))\leq cdn^{1.5}\log n,d\geq0.5\\
	&master\ case3:T(n)=\Theta (n^{\frac{3}{2}}\log n)
	\end{aligned}
	$
	\subsubsection*{(8)}
	$\begin{aligned}
	T(n)
	=T(n-1)+2
	=T(n-2)+4
	=\cdots
	=T(1)+2(n-1)
	=\Theta (n)
	\end{aligned}
	$
	\subsubsection*{(9)}
	$\begin{aligned}
	T(n)
	&=T(n-1)+n^c
	=T(n-2)+n^c+(n-1)^c\\
	&=\cdots
	=T(1)+n^c+(n-1)^c+\cdots+2^c+1,\ c\geq1\\
	&=\Theta (n^{c+1})
	\end{aligned}
	$
	\subsubsection*{(10)}
	$\begin{aligned}
	T(n)
	&=T(n-1)+c^n
	=T(n-2)+c^n+c^(n-1)\\
	&=\cdots
	=T(1)+c^n+c^{n-1}+\cdots+c^2+c,\ c\geq1,\ geometric\ series\\
	&=\Theta (c^n)
	\end{aligned}
	$
	\subsubsection*{(11)}
	$T(n)=T(n/2)+T(n/4)+T(n/8)+n$递归树,每一层的$f(n)=(\frac{7}{8})^h n$,h为结点所在树的高度(根高度记为0),树高为$k=\Theta (\log n)$,则,
	$$T(n)=n+\frac{7}{8}n+\cdots+(\frac{7}{8})^k n=\Theta (n)$$
	\subsection*{problem 2.17}
	\markright{problem 2.17}
	\addcontentsline{toc}{subsection}{problem 2.17}
	\subsubsection*{化简}
	$\begin{aligned}
	&T(n)=\sqrt{n}T(\sqrt{n})+n\\
	&\frac{T(n)}{n}=\frac{T(\sqrt{n})}{\sqrt{n}}+1\\
	&A(n)=\frac{T(n)}{n}
	\end{aligned}
	$
	\subsubsection*{A(n)}
	$\begin{aligned}
	A(n)&=A(\sqrt{n})+1\\
	&=A(\sqrt[4]{n})+2\\
	&=\cdots\\
	&=A(1)+k,n=2^{2^k},k=\log\log n\\
	&=\Theta(\log\log n)\\
	\end{aligned}
	$\\
	\subsubsection*{解得}
	$T(n)=\Theta(n\log\log n)$
	\subsection*{problem 2.18}
	\markright{problem 2.18}
	\addcontentsline{toc}{subsection}{problem 2.18}
	$\begin{aligned}
	&a=1,b=2,f(n)=c\log n^{log_2 1}=0, three\ conditions\ not\ statisfied\\
	&case\ 1:\ \forall \varepsilon >0,\ f(n)\neq O(n^{\log_2 1-\varepsilon})\\
	&case\ 2:\ \ \ \ \ \ \ \ \ \ \ \ f(n)\neq\Theta(n^{\log_2 1})\\
	&case\ 3:\ \forall \varepsilon >0,\ f(n)\neq\Omega(n^{\log_2 1+\varepsilon})\\
	\end{aligned}
	$
	\newpage
	\subsection*{problem 2.20}
	\markright{problem 2.20}
	\addcontentsline{toc}{subsection}{problem 2.20}
	\subsubsection*{MYSTERY}
	\paragraph{result:}
	$r= \sum_{i=1}^{n-1}\sum_{j=i+1}^{n}\sum_{k=1}^{j} 1=\frac{n^3}{3}-\frac{n}{3}$
	\paragraph{最坏时间复杂度:} $T(n)=O(n^3)$
	\subsubsection*{PERSKY}
	\paragraph{result:}
	$r= \sum_{i=1}^{n}\sum_{j=1}^{i}\sum_{k=j}^{i+j} 1=\frac{n^3}{3}+n^2+\frac{2n}{3}$
	\paragraph{最坏时间复杂度:} $T(n)=O(n^3)$
	\subsubsection*{PRESTIFEROUS}
	\paragraph{result:}
	$r= \sum_{i=1}^{n}\sum_{j=1}^{i}\sum_{k=j}^{i+j}\sum_{l=1}^{i+j-k} 1=\frac{n^4}{8}+\frac{5n^3}{12}+\frac{3n^2}{8}+\frac{n}{12}$
	\paragraph{最坏时间复杂度:} $T(n)=O(n^4)$
	\subsubsection*{CONUNDRUM}
	\paragraph{result:}
	$r= \sum_{i=1}^{n}\sum_{j=i+1}^{n}\sum_{k=i+j-1}^{n} 1=\frac{n^2}{2}-\frac{n}{2}$
	\paragraph{最坏时间复杂度:} $T(n)=O(n^2)$
	\newpage
	\section*{PART III}
	\markright{PART III}
	\addcontentsline{toc}{section}{PART III}
	\subsection*{problem 3.2}
	\markright{problem 3.2}
	\addcontentsline{toc}{subsection}{problem 3.2}
	\subsubsection*{(1)}
	\paragraph{对外层循环进行数学归纳:} 
	\subparagraph{Base Case}
	外层循环第一轮结束后,使得最后一个元素大于前n-1个元素。
	外层循环第二轮结束后,使得倒数第二个元素大于前n-2个元素,且满足最后两个元素有序。
	\subparagraph{Assumption}
	假设外层循环第k轮结束后,使得倒数第k个元素大于前n-k个元素,且后k个元素有序即$A[n-k+1]<A[n-k+2]<\cdots<A[n]$。
	\subparagraph{Induction}
	当外层循环第k+1轮结束后,由于$A[1\cdots n-k-1]<A[n-k]<A[n-k+1]$,且后k个元素有序,则$A[n-k]<A[n-k+1]<\cdots<A[n]$后k+1个元素有序。进行最后一轮循环后,n个元素有序,证毕。
	\subsubsection*{(2)}
	\paragraph{最坏时间复杂度与平均复杂度相同:}
	任何情况下,内层循环都要进行$i-1$次比较,最坏与平均情况时间复杂度均为$\sum_{i=n}^2\sum_{j=1}^{i-1}1=\frac{n^2}{2}-\frac{n}{2}=\Theta(n^2)$。
	\subsubsection*{(3)}
	\paragraph{最坏时间复杂度:}不影响最坏情况时间复杂度,最坏情况下,原始序列是降序的,每次内层循环的最后一次交换都发生在待排序序列的末尾,因而每次交换次数仍未$i-1$次,时间复杂度不变。
	\paragraph{平均时间复杂度:}影响平均情况最坏时间复杂度,改进后内层循环的最后一次交换可能发生在待排段的中间位置,使得内层循环实际运行次数小于$i-1$,交换和比较次数减少,但是不影响其量级仍为$\Theta(n^2)$。
	\newpage
	\subsection*{problem 3.4}
	\markright{problem 3.4}
	\addcontentsline{toc}{subsection}{problem 3.4}
	\begin{algorithm}
		\caption{PREVIOUS-LARGER改进算法}
		\label{previous}
		\begin{algorithmic}[1]
			\Function {:PREVIOUS-LARGER}{$A[1\cdots n]$}
			\For {$i\gets1\ \ to\ n$}
			\State $j \gets i-1$
			\While{$j>0 $ \textbf{and} $A[j]\leq A[i]$}
			\State $j \gets p[j]$
			\EndWhile
			\State $p[i]\gets j$
			\EndFor
			\State \Return{$p[1\cdots n]$}
			\EndFunction
		\end{algorithmic}
	\end{algorithm}
	\paragraph{证明 Induction on i:}\label{Induction_previous}
	\begin{itemize}
		\item \textbf{Base case}
		\begin{itemize}
			\item {$i=1:p[1]=0$,正确。}
		\end{itemize}
		\item \textbf{Assumption}
		\begin{itemize}
			\item {$i=k:\ p[1\cdots k]$均已被正确得到。}
		\end{itemize}
		\item \textbf{Induction}
		\begin{itemize}
			\item {$i=k+1:$在While循环体中,当$A[j]\leq A[k+1]$时,$j = p[j]$,又由于$A[p[j]+1\cdots (j-1)]\leq A[j]$,则$p[k+1] \leq p[j]$。}
			\item[]{内层循环中重复向左移动j指针,直到退出循环,在退出前一次循环有$A[j]\leq A[k+1]\ and\ A[p[j]]\geq A[k+1] \ and \ p[k+1] \leq p[j]$,则$p[k+1]=p[j]$正确,算法正确性证毕。}
		\end{itemize}
	\end{itemize}
	\paragraph{时间复杂度}
	最外层循环遍历n次,内层循环每次向前移动常数次,因此时间复杂度为$\Theta(n)$。
	\newpage
	\subsection*{problem 3.5}
	\markright{problem 3.5}
	\addcontentsline{toc}{subsection}{problem 3.5}
	\begin{algorithm}
		\caption{颠倒单词顺序}
		\label{reverse}
		\begin{algorithmic}[1]
			\Require 待翻转的句子S,句子长度L
			\Ensure 翻转后的句子S
			\Function {REVERSE\_STR}{$S,start,end$}\\
			\verb|/**|首先翻转整个句子的每个字符\verb|**/|
			\State $low \gets start,\ high \gets end$
			\While {$low < high$}
			\State $swap(S[low],S[high])$ \verb|\\|句子头尾字符进行翻转
			\State $low++,\ high--$
			\EndWhile
			\EndFunction
			\Function {REVERSE\_WORD}{$S,L$}\\
			\verb|/**|然后翻转句子中的每个单词\verb|**/|
			\State $start \gets 0,\ end \gets 0$
			\For {$i \gets 0\ to\ L$}
			\State $end\gets i$
			\If {$S[end]=blank$}\verb|\\|单词范围为[start,end-1]
			\State $SWAP\_STR(S,start,end-1)$\verb|\\|对这个单词进行翻转
			\State $start \gets end+1$ 
			\EndIf
			\EndFor
			\State \Return {$S$}
			\EndFunction
		\end{algorithmic}
	\end{algorithm}
	
	\paragraph{时间复杂度}首先翻转整个句子前后指针同时移动$\frac{n}{2}$次,之后遍历整个句子的每一个单词的分割移动$n$次,因此时间复杂度为$\frac{n}{2}+n=\frac{3n}{2}=O(n)$。
	\paragraph{空间复杂度}除了最开始的原始输入字符串外,只使用若干常数级的变量,因此如果不考虑输入字符串空间复杂度为$O(1)$。
	\newpage
	\subsection*{problem 3.6}
	\markright{problem 3.6}
	\addcontentsline{toc}{subsection}{problem 3.6}
	\paragraph{(1)}
	可能有1个名人或者没有名人。
	\paragraph{(2)}
	\begin{algorithm}
		\caption{寻找名人}
		\label{find}
		\begin{algorithmic}[1]
			\Require 人群集合$S[1\cdots n]$,人数n
			\Ensure 名人target
			\Function {FIND\_TARGET}{$S,n$}\\
			\verb|/**|如果A关注B则A不是名人,如果B关注A则B不是名人,\\
			如果AB互相关注则均不是名人,如果AB互不关注均不是名人
			\verb|**/|
			\State $num \gets n$
			\State $first \gets 0,\ next \gets 1$
			\While {$n>0$}
			\If {$S[first] \ YES\ S[next]$ \textbf{and} $S[next] \ NO\ S[first]$}
			\State $first \gets next,\ next \gets next+1$ \verb|\\|保证first指向可能的名人
			\State{$n \gets n-1$}
			\ElsIf {$S[first] \ NO\ S[next]$ \textbf{and} $S[next] \ YES\ S[first]$}
			\State $next \gets next+1 $
			\State{$n \gets n-1$}
			\Else
			\State $first \gets next+1,\ next \gets next+2 $
			\State{$n\gets n-2$}
			\EndIf
			\EndWhile\\
			\verb|/**|如果first为名人则一定在1-n中,否则退出循环体时first大于n\verb|**/|
			\State \Return {$first \leq num ? first :0$}\verb|  \\|0表示没有名人
			\EndFunction
		\end{algorithmic}
	\end{algorithm}
	\textbf{时间复杂度}遍历一遍所有人群即可,时间复杂度为$\Theta(n)$。
	\newpage
	\subsection*{problem 3.7}
	\markright{problem 3.7 算法(1)}
	\addcontentsline{toc}{subsection}{problem 3.7}
	\begin{algorithm}
		\caption{(1) 最大连续子序列和暴力算法$O(n^3)$}
		\label{brute force one}
		\begin{algorithmic}[1]
			\Function {MAX\_SUM1}{$S$}\\
			\verb|/**|枚举所有子序列,并根据子序列上下界,计算其子序列和,与当前最大值进行比较\verb|**/|
			\State $res\gets0,len \gets S.length()$
			\For {$i \gets 1 \ to\ len$}
			\For {$j \gets 0 \ to\ i-1$}
			\State $cur\_res \gets 0$
			\For {$k \gets i \ to \ j$}
			\State $cur\_res \gets cur\_res +S[k]$
			\EndFor
			\State $res \gets max(cur\_res,res)$
			\EndFor
			\EndFor
			\State \Return {$res$}
			\EndFunction
		\end{algorithmic}
	\end{algorithm}
	\newpage
	\markright{problem 3.7 算法(2)}
	\begin{algorithm}
		\caption{(2) 最大连续子序列和暴力算法$O(n^2)$}
		\label{brute force two}
		\begin{algorithmic}[1]
			\Function {MAX\_SUM2}{$S$}\\
			\verb|/**|枚举所有子序列,数组SUM[i]记录前i项数据和,利用SUM数组根据子序列头尾计算当前子序列和,与当前最大值进行比较\verb|**/|
			\State $res\gets0,len \gets S.length()$
			\For {$i \gets 2 \ to\ len$}
			\State $sum[i]=sum[i-1]+S[i]$
			\For {$j \gets 1 \ to\ i-1$}
			\State $res \gets max(sum[i]-sum[j-1],res)$
			\EndFor
			\EndFor
			\State \Return {$res$}
			\EndFunction
		\end{algorithmic}
	\end{algorithm}
\newpage
\markright{problem 3.7 算法(3)}
	\begin{algorithm}
		\caption{(3) 最大连续子序列和分治算法$O(n\log n)$}
		\label{conquer}
		\begin{algorithmic}[1]
			\Function {MAX\_SUM3}{$S,start,end$}\\
			\verb|/**|最大子序列和的位置存在三种情况:1.在左半部分,2.在右半部分,3.跨越左右两部分,三者中的最大者即为所求。\verb|**/|
			\State $res\gets0,len \gets S.length(),\ sum[1]=S[1]$
			\If {$start=end$}
			\State $res=S[start]>0?S[start]:0$
			\Else
			\State $center \gets (start+end)/2$
			\State $left\_res=MAX\_SUM3(S,start,center)$
			\State $right\_res=MAX\_SUM3(S,center+1,end)$\\
			\verb|/**|计算包含左半部分最右元素的最大和s1\verb|**/|
			\State $left\gets 0,\ s1\gets 0$
			\For {$i \gets center \ to\ start$}
			\State $left\gets lefts+S[i]$
			\State $s1\gets max(left,s1)$\\
			\EndFor
			\verb|/**|同理计算包含右半部分最左元素的最大和s2	\verb|**/|
			\State $res=max(left\_res,right\_res,s1+s2)$
			\EndIf
			\State \Return {$res$}
			\EndFunction
		\end{algorithmic}
	\end{algorithm}
\newpage
\markright{problem 3.7 算法(4)(5)}
	\begin{algorithm}
		\caption{(4) 最大连续子序列和去冗余算法}
		\label{cut redundancy}
		\begin{algorithmic}[1]
			\Function {MAX\_SUM4}{$S$}\\
			\verb|/**|如果当前子序列和非正,那么对后续子序列和无贡献应当舍去\verb|**/|
			\State $res\gets0,len \gets S.length(),\ sum\gets0$
			\For {$i \gets 0 \ to\ len$}
			\State $sum=sum>0?sum+S[i]:S[i]$
			\State $res=max(sum,res)$
			\EndFor
			\State \Return {$res$}
			\EndFunction
		\end{algorithmic}
	\end{algorithm}

	\begin{algorithm}
		\caption{(5) 最大连续子序列和动态规划算法}
		\label{dynamic program}
		\begin{algorithmic}[1]
			\Function {MAX\_SUM4}{$S$}\\
			\verb|/**|sum[i]用来记录以S[i]结尾的序列最大和\verb|**/|
			\State $res\gets0,len \gets S.length(),\ sum[0]\gets0$
			\For {$i \gets 1 \ to\ len$}
			\State $sum[i]=max(sum[i],sum[i-1]+num[i])$
			\State $res=max(sum[i],res)$
			\EndFor
			\State \Return {$res$}
			\EndFunction
		\end{algorithmic}
	\end{algorithm}
\end{document}