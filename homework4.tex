\documentclass[11pt,a4paper,oneside,oldfontcommands]{ctexart}
\usepackage{tabularx}
\usepackage{array}
\usepackage{bm}
\usepackage{hyperref}
\usepackage{graphicx}
\usepackage{amsmath}
\usepackage{algorithm}
\usepackage{algpseudocode}
\usepackage{fancyhdr}
\pagestyle{fancy}

\hypersetup{hypertex=true,
	colorlinks=true,
	linkcolor=red,
	anchorcolor=blue,
	citecolor=blue}
\fancyhf{}
\chead{\textbf{算法设计与分析}}
\fancyhead[r]{\bfseries\thepage}
\fancyhead[l]{\bfseries\rightmark}
\renewcommand{\headrulewidth}{0.4pt} % 注意不用 \setlength
\renewcommand{\footrulewidth}{0pt}

\floatname{algorithm}{算法}
\renewcommand{\algorithmicrequire}{\textbf{输入:}}
\renewcommand{\algorithmicensure}{\textbf{输出:}}

\begin{titlepage}
	\title{\Huge\textbf{算法设计与分析 作业四}\\}
	\author{\Large\textbf{作者}:吴润泽 \and{\Large\textbf{学号}:181860109}\\
	\\
	\and {\Large\textbf{Email}:\href{mailto:181860109@smail.nju.edu.cn}{181860109@smail.nju.edu.cn}}\\}
	\date{\Large\today}
\end{titlepage}
\begin{document}
\maketitle
\newpage
\tableofcontents
\cleardoublepage
\section*{Chapter 4}
\markright{Chapter 4}
\addcontentsline{toc}{section}{Chapter 4}
{\subsection*{problem 4.2}}
\markright{problem 4.2}
\addcontentsline{toc}{subsection}{problem 4.2}
\hypertarget{(1)}{\subsubsection*{(1)}}
\textbf{$\Rightarrow$}如果w是v在DFS树中的后继结点,那么$actice(w)\subseteq active(v)$:\\
当$w\ne v$时,
因为w是v的后继结点,所以$v.discover<w.discover$,并且$v.finish>w.finish$。所以$actice(w)\subset active(v)$。当$w=v$时显然成立。
\textbf{$\Leftarrow$}如果$actice(w)\subseteq active(v)$,那么w是v在DFS树中的后继结点:\\
当$w\ne v$时,因为$active(w)\subseteq active(v)$,即$v.discover<w.discover$,并且$v.finish>w.finish$,即在遍历v的过程中将w遍历,即w是v的后继结点。
\hypertarget{(2)}{\subsubsection*{(2)}}
由\hyperref[(1)]{(1)}可知,w不是v的后继结点$\Leftrightarrow actice(w)\not\subset active(v)$。
v不是w的后继结点$\Leftrightarrow actice(v)\not\subset active(w)$得证。
\subsubsection*{(3)}
\paragraph*{\textcircled{1}}
\textbf{$\Rightarrow$}如果vw是CE,那么v和w没有祖先和后继关系,
由\hyperref[(2)]{(2)}可知active(w)和active(v)互不包含。同时CE说明在v指向w时,w已经是黑色结点,w已经遍历结束,
所以active(w)在active(v)之前。\\
\hspace*{20pt}\textbf{$\Leftarrow$ }active(w)在active(v)之前,w先完成整个遍历过程,后才遍历到v。且二者没有祖先后继关系,
那么边vw即为CE。\\
\paragraph*{\textcircled{2}}
\textbf{$\Rightarrow$}vw是DE,即v指向w时w为黑色,并且$active(w)\subset active(v)$,
若不存在第三个结点x,满足x是v的后继,w是x的后继,则v遍历到w时w一定为白色,边为TE。
即一定有$active(w)\subset active(x)\subset active(v)$。\\
\hspace*{20pt}\textbf{$\Leftarrow$}如果存在结点x,满足$active(w)\subset active(x)\subset active(v)$,
由\hyperref[(1)]{(1)}可知,x是v的后继,w是x的后继,且在遍历时v先走到x,然后x走到w,即v是w的祖先结点,因此vw是DE。\\
\paragraph*{\textcircled{3}}
\textbf{$\Rightarrow$}vw是TE,即v指向w时w为白色,w是v的后继,由\hyperref[(1)]{(1)}可知,$active(w)\subset active(v)$。
若存在x,满足$active(w)\subset active(x)\subset active(v)$,则在遍历时v先走到x,然后x走到w,
那么v是w的祖先而非父结点,与vw是TE矛盾。\\
\hspace*{20pt}\textbf{$\Leftarrow$}同理可得w是v的后继,且v直接指向w,则w是白色,即vw是TE。\\
\paragraph*{\textcircled{4}}
vw是BE$\Leftrightarrow$v是w的后继$\Leftrightarrow active(v)\subset active(w)$得证。\\

{\subsection*{problem 4.5}}
\markright{problem 4.5}
\addcontentsline{toc}{subsection}{problem 4.5}
\noindent \hypertarget{1.}{1. }不可能是TE。如果是TE,则有$active(v)\subset actice(u)$,即$v.finishTime>u.discoverTime$,故不成立。\\
2. 不可能是BE。如果是BE,则有$active(u)\subset actice(v)$,即$v.finishTime>u.discoverTime$,故不成立。\\
3. 不可能是DE。如果是DE,同样的v是u的后继结点,满足$active(v)\subset actice(u)$,同\hyperref[1.]{1.}不成立。\\
4. 可能是CE。
x结点先遍历v,然后从v返回x,x遍历u,易知$v.finishTime<u.discoverTime$。\\
如图所示\includegraphics{CEORDER.png}\\
\newpage
{\subsection*{problem 4.7}}
\markright{problem 4.7}
\addcontentsline{toc}{subsection}{problem 4.7}
在第一次DFS中,将结点压栈,同一强连通片的源头结点是最后一个压栈的。\textbf{(引理4.4)}若l是某个强连通片首结点,x是另一个强连通片中的结点,
并且存在l通向x的路径,则x比l先结束遍历,即x先进栈l后进栈。满足这些性质,才能保证第二次DFS中,
按正确的顺序取出每个SCC的首结点。\\
\hspace*{20pt}无论是DFS还是BFS在一次遍历中都可以访问一个或者多个强连通片的所有结点。\\
\hspace*{20pt}如果第一次DFS换为BFS,由于BFS中按层序遍历,同一连通片中出度不为0的点可能先入栈。在第二次DFS的时候,不能有正确的访问顺序。
\hspace*{20pt}如果第二次DFS换为BFS,由于出栈的访问首结点顺序正确,BFS同样可以正确地划分强连通片。\\
\hspace*{20pt}因此第一次必须为DFS,第二次DFS和BFS都可以。
{\subsection*{problem 4.8}}
\markright{problem 4.8}
\addcontentsline{toc}{subsection}{problem 4.8}
\noindent\textbf{充要条件:}对于无向连通图的DFS生成树的根结点v,v是割点,当且仅当v有两个及两个以上的子树。
\paragraph{证明:}
\textbf{$\Rightarrow$}如果v是割点,假设v只有一个子树,易知子树是连通的,将v删除,
剩下的部分为v的子树仍然连通,这与v是割点相矛盾。因此v有两个及两个以上的子树。\\
\textbf{$\Leftarrow$}如果v有两个及两个以上的子树,因为图本身连通,则子树之间相连必然通过v,即v是割点。
{\subsection*{problem 4.9}}
\markright{problem 4.9}
\addcontentsline{toc}{subsection}{problem 4.9}
正确
\paragraph{证明:}
当从TE vw回退时,如果以w为根的子树存在BE指向v的祖先,
则v的祖先的discoverTime会被赋值给以w为根的子树中某个结点的back值,
并最终会传递到w.back,即必有$w.back<v.discoverTime$;\\
\hspace*{20pt}否则,没有存在BE指向v的祖先,如果子树不存在BE,则子树所有点back值均为初始值,
即$w.back>v.discoverTime$,如果子树中存在BE,则子树中的所有结点的back值也将大于等于$v.discoverTime$,
因为BE只能指向v或者子树内部结点,故back值最小也不会低于$v.discoverTime$,仍然满足$w.back>v.discoverTime$,即仍能正确判断是否为割点。
{\subsection*{problem 4.12}}
\markright{problem 4.12}
\addcontentsline{toc}{subsection}{problem 4.12}
算法中,默认环为多条TE和BE构成,但存在两条BE和TE构成环的情况。从A点出发,依次遍历B,C,D,E。
如下图所示。\\
\includegraphics{4-12.png}\\
图中的BE为AE和AD,根据题目给定算法,计算出外圈大环的大小为5和右侧环大小为4,最终结果为4。
但图中最小环大小应为3,为左侧小环。因此算法不正确。
{\subsection*{problem 4.13}}
\markright{problem 4.13}
\addcontentsline{toc}{subsection}{problem 4.13}
如果有向图没有环,则至少有一个点的入度为0。即无向图DFS生成树必须存在环。
\paragraph{算法设计}
1. 有向图中每个点的入度至少为1,则有图的边数不小于图的点数,则无向图中必定有环(存在BE)。\\
2. 利用DFS找到一条BE(找到一个环),设边为uv,u为后继,v为祖先。
(如果没有找到BE,说明存在一棵DFS生成树无环,则不能构造成功)\\
3. 以v为起点进行DFS,将每条TE xy,标记方向为x->y,最后将uv边标记方向为u->v,即满足了该环中每个点的入度都大于0。\\
4. 找到图中仍为白色的点,进行2、3步,若没有算法结束。\\
5. 由于只进行了两次深度优先遍历,较易得算法时间复杂度为线性。\\
\textbf{具体算法实现见}\hyperref[AddDirection]{AddDirection算法}\\
$\begin{aligned}
		\hline
		\label{AddDirection}
		 & \textbf{算法1 }AddDirection\text{ 算法}                                                             \\
		\hline
		 & 1.\textbf{Function}\ {FindLoop}\ (u)  \verb|\\|\text{找到一个环路,并返回BE边祖先结点} \\
		 & 2.\hspace*{20pt}u.color:=GRAY, res:=None                                                            \\
		 & 3.\hspace*{20pt}\textbf{foreach }neighbor\ v\ of\ u\textbf{ do}                                     \\
		 & 4.\hspace*{40pt}\textbf{if }v.color=WHITE\textbf{ and }v.vis=False\textbf{ then}                    \\
		 & 5.\hspace*{60pt}FindLoop(v)                                                                         \\
		 & 6.\hspace*{40pt}\textbf{elif }uv=BE\textbf{ then}                                                   \\
		 & 7.\hspace*{60pt}res:=u,\textbf{break}                                                               \\
		 & 8.\hspace*{20pt}\textbf{return }res                                                                 \\
		\hline
		 & 1.\textbf{Function}\ {AddDir}\ (u)    \verb|\\|\text{给环路加方向}                     \\
		 & 2.\hspace*{20pt}u.vis=True\verb|\\|\text{标记其已经加边避免重复}                       \\
		 & 3.\hspace*{20pt}\textbf{foreach }neighbor\ v\ of\ u\textbf{ do}                                     \\
		 & 4.\hspace*{40pt}\textbf{if }v.color=WHITE\textbf{ and }v.vis=False\textbf{ then}                    \\
		 & 5.\hspace*{60pt}change\ uv\ dircetion\ to\ u\rightarrow v                                           \\
		 & 6.\hspace*{60pt}AddDir(v)                                                                           \\
		 & 7.\hspace*{40pt}\textbf{elif }uv=BE\textbf{ then}                                                   \\
		 & 8.\hspace*{60pt}change\ uv\ dircetion\ to\ u\rightarrow v                                           \\
		\hline
		 & 1.\textbf{Function}\ {Main-AddDirection}\ (V,E)   \verb|\\|\text{wrapper部分}          \\
		 & 2.\hspace*{20pt}\textbf{foreach }point\ v\ in\ V\textbf{ do}                                        \\
		 & 3.\hspace*{40pt}v.color:=WHITE,v.vis:=False                                                         \\
		 & 4.\hspace*{20pt}\textbf{foreach }point\ v\ in\ V\textbf{ do}                                        \\
		 & 5.\hspace*{40pt}\textbf{if }v.color=WHITE\textbf{ and }v.vis=False\textbf{ then}                    \\
		 & 6.\hspace*{60pt}res:=FindLoop(v)                                                                    \\
		 & 7.\hspace*{60pt}\textbf{if }res=None\textbf{ then }\textbf{return }False                            \\
		 & 9.\hspace*{60pt}\textbf{else }AddDir(res)                                                           \\
		 & 10.\hspace*{15pt}\textbf{return }True                                                               \\
		\hline
	\end{aligned}
$\newpage
{\subsection*{problem 4.14}}
\markright{problem 4.14}
\addcontentsline{toc}{subsection}{problem 4.14}
\paragraph{算法设计}
因为G是有向无环图,则G一定存在拓扑排序(\textbf{引理4.2})。\\
1. 入度最小的点(即拓扑排序的队首)a,如果图中存在哈密顿路径,则a的入度一定为0。假设a的入度不为0,结点x指向a。
由于a是拓扑排序队首,故一定存在a到达x的路径(因为有图连通)设为$a\rightarrow\cdots\rightarrow x$,同时x指向a,构成环路,与题干相矛盾。故a的入度一定为0。\\
2. 显然,从a点出发,如果存在哈密顿路径,则一次DFS遍历所有结点。\\
3. 首先找到图G的入度为0的点a,如果不存在或存在多个则不可能存在哈密顿通路。\\
4. 从a开始DFS,每当递归回溯时(边为uv),检查哈密顿路径path长度是否为结点个数。\\
5. 如果等于结点个数,说明存在路径$a\rightarrow\cdots u\rightarrow v\rightarrow\cdots n$,算法结束。\\
6. 否则,将$v.vis$置为假,将v从path中剔除(哈密顿通路上u不直接到达v),从u的下一子结点继续DFS。\\
7. 找最小入度结点为$O(m+n)$,DFS过程显然是$O(m+n)$,因此时间复杂度是线性的。\\
\textbf{具体算法实现见}\hyperref[FindHamilton]{FindHamilton算法}\\
$\begin{aligned}
		\hline
		\label{FindHamilton}
		 & \textbf{算法2 }FindHamilton \text{ 算法}                                                                       \\
		\hline
		 & 1.\textbf{Function}\ {FindHamilton}\ (u,path,n)  \verb|\\|\text{找到一个环路,并返回BE边祖先结点} \\
		 & 2.\hspace*{20pt}u.vis:=True,path.push(u)                                                                       \\
		 & 3.\hspace*{20pt}\textbf{foreach }neighbor\ v\ of\ u\textbf{ do}                                                \\
		 & 4.\hspace*{40pt}\textbf{if }v.vis=False\textbf{ then}                                                          \\
		 & 5.\hspace*{60pt}FindHamilton(v,path,n)                                                                         \\
		 & 5.\hspace*{60pt}\textbf{if }len(path)==n\textbf{ then return }True                                             \\
		 & 6.\hspace*{60pt}v.vis:=False,path.pop()\verb|\\|\text{没有找到,故将vis置为假,将v从path剔除}     \\
		 & 7.\hspace*{20pt}\textbf{return }False\verb|\\|\text{u所有子结点都没有找到哈密顿通路}              \\
		\hline
	\end{aligned}
$
$\begin{aligned}
		\hline
		 & 1.\textbf{Function}\ {Main-FindHamilton}\ (V,E)   \verb|\\|\text{wrapper部分}                                           \\
		 & 2.\hspace*{20pt}\textbf{foreach }point\ u\ in\ V\textbf{ do}                                                                         \\
		 & 3.\hspace*{40pt}u.vis:=False,u.cnt:=0\verb|\\|\text{入度初始化为0}                                                      \\
		 & 4.\hspace*{40pt}\textbf{foreach }neighbor\ v\ of\ u\textbf{ do }u.cnt++\verb|\\|\text{计算各点的入度}                  \\
		 & 5.\hspace*{20pt}tar:=None,cnt:=0  \verb|\\|\text{初始化要寻找的点}                                                     \\
		 & 6.\hspace*{20pt}\textbf{foreach }point\ u\ in\ V\textbf{ do}     \verb|\\|\text{找到唯一一个入度为0的点}               \\
		 & 7.\hspace*{40pt}\textbf{if }tar!=None\textbf{ and }cnt=0\textbf{ and }u.cnt=0                                                        \\
		 & 8.\hspace*{60pt}\textbf{ then return } False\verb|\\|\text{存在不止一个入度为0的点,不可能存在哈密顿通路}              \\
		 & 9.\hspace*{40pt}\textbf{if }cnt\geq u.cnt\textbf{ then }tar:=u,cnt:=u.cnt                                                            \\
		 & 10.\hspace*{15pt}\textbf{if }tar=None\textbf{ or }cnt!=0\textbf{ then return } False \verb|\\|\text{不存在入度为0的点} \\
		 & 11.\hspace*{15pt}\textbf{return }FindHamilton(tar)                                                                                   \\
		\hline
	\end{aligned}
$

\markright{problem 4.16}
\hypertarget{problem 4.16}{\subsection*{problem 4.16}}
\addcontentsline{toc}{subsection}{problem 4.16}
\noindent \textbf{证明 }有向无环图中必有入度为0的点。\\
\hspace*{20pt}设图有N个结点,假设每个点的入度均不为0,必有
$A_{2}\rightarrow A_{1},A_{3}\rightarrow A_{2},\cdots,A_{n}\rightarrow A_{n-1}$,
而对于$A_n$入度不为0,故存在某个结点指向它,则出现环路,产生矛盾,得证。\\
\textbf{算法设计}\\
\textbf{1.} 有向无环图可以出现一个或者多个入度为0的结点,对这些结点的拓扑先后顺序没有要求。\\
\textbf{2.} 入度为0的结点的出边删去,必会出现入度为0的点(仍为无环图)。\\
\textbf{3.} 重复\textbf{1,2}步,直到算法结束。可知时间复杂度为线性。\\
由开始的\textbf{证明}可知,图中存在回路,会出现没有入度为0的点的情况,无法确定拓扑顺序。\\
\textbf{算法实现}\\
$\begin{aligned}
		\hline
		 & \textbf{算法3 }ZeroTopoLogical \text{ 算法}                                                                      \\
		\hline
		 & 1.\textbf{Function}\ {InitQueue}\ (G(V,E),queue)  \verb|\\|\text{找到一个环路,并返回BE边祖先结点} \\
		 & 2.\hspace*{20pt}\textbf{foreach }point\ u\ in\ V\textbf{ do}                                                     \\
		 & 3.\hspace*{40pt}\textbf{if }InEdge[u]=0\textbf{ then }                                                           \\
		 & 4.\hspace*{60pt}queue.push(u)                                                                                    \\
		 & 5.\hspace{20pt}\textbf{return }len(queue)>0\verb|\\|\text{没有找到入度为0的点,返回False}          \\
		\hline
	\end{aligned}
$
$\begin{aligned}                                                                \\
		\hline
		\label{FindHamiltZeroTopoLogical}
		 & 1.\textbf{Function}\ {Main-ZeroTopoLogical}\ (G(V,E))  \verb|\\|\text{找到一个环路,并返回BE边祖先结点}                        \\
		 & 2.\hspace*{20pt}queue.init(),topo:=list(),count:=0                                                                                           \\
		 & 3.\hspace*{20pt}\verb|/*|\text{queue存放入度为0的结点的队列,topo存放拓扑排序结果,count为排好序的结点数}\verb|*/| \\
		 & 4.\hspace*{20pt}\textbf{if }InitQueue(queue)=False\textbf{ then return} False                                                                 \\
		 & 5.\hspace*{20pt}\textbf{while }queue.empty()!=False\textbf{ do}                                                                              \\
		 & 6.\hspace*{40pt}u=queue.pop(),topo[count++]=u\verb|\\|\text{确定u为当前的逻辑起点}                                             \\
		 & 7.\hspace*{40pt}\textbf{foreach }neighbor\ v\ of\ u\textbf{ do}                                                                              \\
		 & 8.\hspace*{60pt}InEdge[v]--                                                                                                                  \\
		 & 9.\hspace*{60pt}\textbf{if }v.vis=False\textbf{ then }queue.push(v)                                                                          \\
		 & 11.\hspace{15pt}\textbf{return }count=len(V)\verb|\\|\text{所有结点应确定拓扑排序,否则存在环路}                              \\
		\hline
	\end{aligned}
$
{\subsection*{problem 4.17}}
\markright{problem 4.17}
\addcontentsline{toc}{subsection}{problem 4.17}
\hypertarget{4.17(1)}{\subsubsection*{(1)}}
\noindent 易知只需要以顶点s为起点进行一次DFS,判断是否遍历所有点即可。代价为$O(m+n)$。\\
$\begin{aligned}
		\hline
		 & 1.\textbf{Function}\ {OneToAll}\ (G(V,E),s)                         \\
		 & 2.\hspace{20pt}DFS(s)                                               \\
		 & 3.\hspace*{20pt}\textbf{foreach }point\ u\ in\ V\textbf{ do}        \\
		 & 4.\hspace{40pt}\textbf{if }u.color=WHITE\textbf{ then return }False \\
		 & 5.\hspace*{20pt}\textbf{return }True                                \\
		\hline
	\end{aligned}$
\subsubsection*{(2)}
利用课本上的划分强连通片的算法,将有向图G改造为G的收缩图G$\downarrow$,
各点之间的方向转换为各连通片之间的方向,易知G$\downarrow$为有向无环图。\\
\hspace*{20pt}题目可以转换为一个连通片可以到达其他所有连通片。
由于G$\downarrow$为有向无环图,必有入度为0的连通片。
故如果图G存在到达所有顶点的点,其必在该连通片中。
再利用\hyperlink{4.17(1)}{(1)}算法判断该连通片能否到达所有连通片即可。\\
\hspace*{20pt}强连通片划分与DFS判断是否可达算法时间复杂度均为$O(m+n)$,因此时间复杂度为$O(m+n)$,符合题意。\\
\textbf{具体算法实现见}\hyperref[SccOneToAll]{SccOneToAll算法}\\
$\begin{aligned}
		\hline
		\label{SccOneToAll}
		 & \textbf{算法4 } \text{SccOneToAll 算法}                                                            \\
		\hline
		 & 1.\textbf{Function}\ {SccOneToAll}\ (G(V,E))                                                       \\
		 & 2.\hspace*{20pt}construct\ G\downarrow\ using\ Scc(G)                                              \\
		 & 3.\hspace*{20pt}point:=None,CountZero:=0\verb|\\|\text{记录入度为0的连通片个数}      \\
		 & 4.\hspace*{20pt}\textbf{foreach }point\ u\ in\ V\downarrow\textbf{ do}                             \\
		 & 5.\hspace*{40pt}\textbf{if }InEdge[u]=0\textbf{ then }                                             \\
		 & 6.\hspace*{60pt}point:=u,CountZero:=CountZero+1                                                    \\
		 & 7.\hspace*{20pt}\textbf{if }point=None\textbf{ or }CountZero>1\textbf{ then }                      \\
		 & 8.\hspace{40pt}\textbf{return }False\verb|\\|\text{没有或多个入度为0的点,返回False} \\
		 & 9.\hspace{20pt}\textbf{return }OneToAll(G\downarrow,point)                                         \\
		\hline
	\end{aligned}
$
{\subsection*{problem 4.18}}
\markright{problem 4.18}
\addcontentsline{toc}{subsection}{problem 4.18}
\subsubsection*{(1)}
同样利用课本上的划分强连通片的算法,将有向图G改造为G的收缩图G$\downarrow$。
易知同一个连通片中的影响力值相同。
易得影响力最低的连通片,一定是出度为0的连通片(一定存在,可能只有一个连通片),影响力是连通片中结点个数减1。
因此,找到所有出度为0的连通片,并找出结点个数最小的即可。\\
\hspace*{20pt}SCC的划分和寻找出度为0的连通片并统计的时间复杂度为$O(m+n)。$\\
$\begin{aligned}
		\hline
		\label{MinImpact}
		 & \textbf{算法5 } \text{MinImpact 算法}                                                                                  \\
		\hline
		 & 1.\textbf{Function}\ {MinImpact}\ (G(V,E))                                                                             \\
		 & 2.\hspace*{20pt}construct\ G\downarrow\ using\ Scc(G)                                                                  \\
		 & 3.\hspace*{20pt}MinPoint:=None,MinCount:=\infty\verb|\\|\text{记录连通片的最小个数}                      \\
		 & 4.\hspace*{20pt}\textbf{foreach }point\ u\ in\ V\downarrow\textbf{ do}                                                 \\
		 & 5.\hspace*{40pt}\textbf{if }OutEdge[u]=0\textbf{ and }MinImpact<number(u)\textbf{ then }                               \\
		 & 6.\hspace*{60pt}MinPoint:=u,MinCount:=number(u)                                                                        \\
		 & 7.\hspace{20pt}\textbf{return }(MinCount-1,MinPoint)\verb|\\|\text{结点个数减1为影响力,以及对应的连通片} \\
		\hline
	\end{aligned}
$
\subsubsection*{(2)}
同样得到有向图G的收缩图G$\downarrow$,易知影响力最大的连通片,一定是入度为0的连通片(压缩图无环)。\\
则对所有的入度为0的连通片进行DFS遍历,遍历到的连通片即为其可到达的,DFS结束后计算遍历到连通片的结点个数。
取所有DFS得到结点个数最多的即为所求。\\
由于每次DFS后,都要查找所有访问过的连通片,因此时间复杂度为$O(n(m+n))$。\\
\textbf{算法实现}\\
$\begin{aligned}
		\hline
		\label{MinImpact}
		 & \textbf{算法6 } \text{MaxImpact 算法}                                                                              \\
		\hline
		 & 1.\textbf{Function}\ {MinImpact}\ (G(V,E))                                                                         \\
		 & 2.\hspace*{20pt}construct\ G\downarrow\ using\ Scc(G)                                                              \\
		 & 3.\hspace*{20pt}MaxPoint:=None,CurCount:=0,MaxCount:=0                                                             \\
		 & 4.\hspace*{20pt}\verb|/*|\text{记录最大的DFS生成树的根以及结点个数}\verb|*/|           \\
		 & 5.\hspace*{20pt}\textbf{foreach }point\ u\ in\ V\downarrow\textbf{ do}                                             \\
		 & 6.\hspace*{40pt}\textbf{if }InEdge[u]=0\textbf{ then }                                                             \\
		 & 7.\hspace*{60pt}DFS(G\downarrow,u),CurCount:=0   \verb|\\|\text{更新当前DFS树的结点个数}             \\
		 & 8.\hspace*{60pt}\textbf{foreach }point\ x\ in\ V\downarrow\textbf{ do}                                             \\
		 & 9.\hspace*{80pt}\textbf{if }x.vis=True\textbf{ then }                                                              \\
		 & 10.\hspace*{95pt}CurCount:=CurCount+number(x)                                                                      \\
		 & 11.\hspace*{95pt}x.vis:=False  \verb|\\|\text{将访问置为False,避免下次DFS错误计算}                  \\
		 & 12.\hspace*{55pt}\textbf{if }MaxCount<CurCount\textbf{ then }                                                      \\
		 & 13.\hspace*{75pt}MaxPoint:=u,MaxCount:=CurCount                                                                    \\
		 & 14.\hspace{15pt}\textbf{return }(MaxCount-1,MaxPoint)\verb|\\|\text{返回最大影响力,以及对应的连通片} \\                                                \\
		\hline
	\end{aligned}
$
\newpage
{\subsection*{problem 4.20}}
\markright{problem 4.20}
\addcontentsline{toc}{subsection}{problem 4.20}
与\hyperlink{problem 4.16}{problem 4.16} 类似,每学期可以修所有入度为0的结点。
然后将所有入度为0的结点的指出的边删去,假设本题有解即为有向无环图,
一定再次出现入度为0的结点,重复上述过程即可修完所有课程,重复的次数即为所需要的最少学期数。
可得算法时间复杂度是线性$O(m+n)$。\\
\textbf{算法实现}\\
$\begin{aligned}
		\hline
		 & \textbf{算法3 }ZeroTopoLogical \text{ 算法}                                                                                               \\
		\hline
		 & 1.\textbf{Function}\ {InitQueue}\ (G(V,E),queue)  \verb|\\|\text{找到一个环路,并返回BE边祖先结点}                          \\
		 & 2.\hspace*{20pt}\textbf{foreach }point\ u\ in\ V\textbf{ do}                                                                              \\
		 & 3.\hspace*{40pt}\textbf{if }InEdge[u]=0\textbf{ and }u.vis=False\textbf{ then }                                                           \\
		 & 4.\hspace*{60pt}queue.push(u)                                                                                                             \\
		 & 5.\hspace*{20pt}\textbf{return }len(queue)>0\verb|\\|\text{没有找到入度为0的点,返回False}                                   \\                                                              \\
		\hline
		\label{FindHamiltZeroTopoLogical}
		 & 1.\textbf{Function}\ {Main-ZeroTopoLogical}\ (G(V,E))  \verb|\\|\text{找到一个环路,并返回BE边祖先结点}                     \\
		 & 2.\hspace*{20pt}queue.init(),count:=0,TermCount:=0                                                                                        \\
		 & 3.\hspace*{20pt}\verb|/*|\text{queue存放入度为0的结点的队列,count为排好序的结点数,TermCount为学期数}\verb|*/| \\
		 & 4.\hspace*{20pt}\textbf{while }count<len(V)\textbf{ do}\verb|\\|\text{安排的课程数还少于总课程}                                                                                   \\
		 & 5.\hspace*{40pt}\textbf{if }InitQueue(queue)=False\textbf{ then return }False                                                             \\
		 & 6.\hspace*{40pt}\textbf{while }queue.empty()!=False\textbf{ do}                                                                           \\
		 & 7.\hspace*{60pt}u=queue.pop()\\
		 & 8.\hspace*{60pt}u.vis=True\\
		 & 9.\hspace*{60pt} count:=count+1\verb|\\|\text{已经确定的课程安排数加1}                              \\
		 & 10.\hspace*{55pt}\textbf{foreach }neighbor\ v\ of\ u\textbf{ do }\\
		 & 11.\hspace*{75pt}InEdge[v]--                                                               \\
		 & 12.\hspace*{35pt}TermCount:=TermCount+1\verb|\\|\text{所有零入度课程均已安排,学期数加1}                       \\
		 & 13.\hspace{15pt}\textbf{return }TermCount\verb|\\|\text{最少学期数}                           \\
		\hline
	\end{aligned}
$
{\subsection*{problem 4.22}}
\markright{problem 4.22}
\addcontentsline{toc}{subsection}{problem 4.22}
{\subsection*{problem 4.23}}
\markright{problem 4.23}
\addcontentsline{toc}{subsection}{problem 4.23}
\section*{Chapter 5}
\markright{Chapter 5}
\addcontentsline{toc}{section}{Chapter 5}
{\subsection*{problem 5.1}}
\markright{problem 5.1}
\addcontentsline{toc}{subsection}{problem 5.1}
{\subsection*{problem 5.2}}
\markright{problem 5.2}
\addcontentsline{toc}{subsection}{problem 5.2}
{\subsection*{problem 5.4}}
\markright{problem 5.4}
\addcontentsline{toc}{subsection}{problem 5.4}
{\subsection*{problem 5.8}}
\markright{problem 5.8}
\addcontentsline{toc}{subsection}{problem 5.8}
{\subsection*{problem 5.9}}
\markright{problem 5.9}
\addcontentsline{toc}{subsection}{problem 5.9}
{\subsection*{problem 5.10}}
\markright{problem 5.10}
\addcontentsline{toc}{subsection}{problem 5.10}
\end{document}