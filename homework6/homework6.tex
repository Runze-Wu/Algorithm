\documentclass[11pt,a4paper,oneside,oldfontcommands]{ctexart}
\usepackage{tabularx}
\usepackage{array}
\usepackage{bm}
\usepackage{hyperref}
\usepackage{graphicx} 
\usepackage{amsmath}
\usepackage{algorithm}
\usepackage{algpseudocode}
\usepackage{fancyhdr}
\pagestyle{fancy}
\usepackage{color}
\usepackage{xcolor}
\definecolor{keywordcolor}{rgb}{0.8,0.1,0.5}
\usepackage{listings}


\definecolor{mygreen}{rgb}{0,0.6,0}  
\definecolor{mygray}{rgb}{0.5,0.5,0.5}  
\definecolor{mymauve}{rgb}{0.58,0,0.82}  
  
\lstset{ %  
  backgroundcolor=\color{white},   % choose the background color; you must add \usepackage{color} or \usepackage{xcolor}  
  basicstyle=\footnotesize,        % the size of the fonts that are used for the code  
  breakatwhitespace=false,         % sets if automatic breaks should only happen at whitespace  
  breaklines=true,                 % sets automatic line breaking  
  captionpos=bl,                    % sets the caption-position to bottom  
  commentstyle=\color{mygreen},    % comment style  
  deletekeywords={...},            % if you want to delete keywords from the given language  
  escapeinside={\%*}{*)},          % if you want to add LaTeX within your code  
  extendedchars=true,              % lets you use non-ASCII characters; for 8-bits encodings only, does not work with UTF-8  
  frame=single,                    % adds a frame around the code  
  keepspaces=true,                 % keeps spaces in text, useful for keeping indentation of code (possibly needs columns=flexible)  
  keywordstyle=\color{blue},       % keyword style  
  %language=Python,                 % the language of the code  
  morekeywords={*,...},            % if you want to add more keywords to the set  
  numbers=left,                    % where to put the line-numbers; possible values are (none, left, right)  
  numbersep=5pt,                   % how far the line-numbers are from the code  
  numberstyle=\tiny\color{mygray}, % the style that is used for the line-numbers  
  rulecolor=\color{black},         % if not set, the frame-color may be changed on line-breaks within not-black text (e.g. comments (green here))  
  showspaces=false,                % show spaces everywhere adding particular underscores; it overrides 'showstringspaces'  
  showstringspaces=false,          % underline spaces within strings only  
  showtabs=false,                  % show tabs within strings adding particular underscores  
  stepnumber=1,                    % the step between two line-numbers. If it's 1, each line will be numbered  
  stringstyle=\color{orange},     % string literal style  
  tabsize=4,                       % sets default tabsize to 2 spaces  
  %title=myPython.py                   % show the filename of files included with \lstinputlisting; also try caption instead of title  
}  
\hypersetup{hypertex=true,
	colorlinks=true,
	linkcolor=red,
	anchorcolor=blue,
	citecolor=blue}
\fancyhf{}
\chead{\textbf{算法设计与分析}}
\fancyhead[r]{\bfseries\thepage}
\fancyhead[l]{\bfseries\rightmark}
\renewcommand{\headrulewidth}{0.4pt} % 注意不用 \setlength
\renewcommand{\footrulewidth}{0pt}

\floatname{algorithm}{算法}
\renewcommand{\algorithmicrequire}{\textbf{输入:}}
\renewcommand{\algorithmicensure}{\textbf{输出:}}

\begin{titlepage}
	\title{\Huge\textbf{算法设计与分析 作业六}\\}
	\author{\Large\textbf{作者}:吴润泽 \and{\Large\textbf{学号}:181860109}\\
	\\
	\and {\Large\textbf{Email}:\href{mailto:181860109@smail.nju.edu.cn}{181860109@smail.nju.edu.cn}}\\}
	\date{\Large\today}
\end{titlepage}
\begin{document}
\maketitle

\tableofcontents
\cleardoublepage
\section*{Chapter 13}
\markright{Chapter 13}
\addcontentsline{toc}{section}{Chapter 13}
{\subsection*{problem 13.1}}
\markright{problem 13.1}
\addcontentsline{toc}{subsection}{problem 13.1}
\subsubsection*{1)}
\begin{lstlisting}[language=C++,title=FloydNext.func]
FloydNext(G):
	D(0):=W /*初始情况下,两点间的最短路径长度就是它们的边权*/
	for(i:=1 to n):
		for(j:=1 to n):
			if map[i][j]!=Inf: /*如果ij间有边连接,则i的后继即为j*/
				GO[i][j]=j
	for(k:=1 to n):
		for(i:=1 to n):
			for(j:=1 to n):
				if D[i][j]>D[i][k]+D[k][j]:
					GO[i][j]=k /*如果i到j通过k的路径更短,则更新路由表*/
					D[i][j]=D[i][k]+D[k][j]
\end{lstlisting}
\paragraph*{2)}
算法更新路径$i \rightarrow k \rightarrow j$,所选择的j的前驱与
从选择的节点k到节点j的一条中间节点取自集合$\{1,2,\cdots,k-1\}$的最短路径上的前驱是一样的。

\begin{lstlisting}[language=C++,title=FloydPrev.func]
FloydPrev(G):
	D(0):=W
	for(i:=1 to n):
		for(j:=1 to n):
			if map[i][j]!=Inf: /*如果ij间有边连接,则j的前驱即为i*/
				GO[i][j]:=i
	for(k:=1 to n):
		for(i:=1 to n):
			for(j:=1 to n):
				if D[i][j]>D[i][k]+D[k][j]:
					GO[i][j]:=GO[k][j] /*如果i到j通过k的路径更短,则前缀为Back[k][j]*/
					D[i][j]:=D[i][k]+D[k][j]
\end{lstlisting}
{\subsection*{problem 13.2}}
\markright{problem 13.2}
\addcontentsline{toc}{subsection}{problem 13.2}
\paragraph*{1)}仿照Dijkstra算法,记录每条路径的最小值即可。
\begin{lstlisting}[language=C++,title=DijkstraCap.func]
DijkstraCap(G):
	initialize the priority queue Fringe as empty
	insert some node v to Fringe
	cap:=[] /*cap 原来存放s到达其他点的吞吐量*/
	while Fringe!=empty:
		v:=Fringe.EXTRACT-MIN()
		cap[v]:=v.priority /*存放到达v的结果*/
		UPDATE-FRINGE(v,Fringe)
	return cap
UPDATE-FRINGE(v,Fringe):
	for neighbor w of v:
		w.priority:=min(vw.weight,v.priority) 
		/*取路径的最小值,即当前路径已知吞吐量*/
		if w is UNSEEN: /*如果是第一次遍历到*/
			Fringe.INSERT(w,w.priority)  /*将w加入队列*/
		else:
			Fringe.decrease(w,w.priority) /*将w的权值更新为w.priority*/
\end{lstlisting}
\paragraph*{2)}仿照Floyd算法,对路径$i \rightarrow k \rightarrow j$从ij,ik,kj中选取最小权边即可。
\begin{lstlisting}[language=C++,title=FloydCap.func]
FloydCap(G):
D(0):=W /*开始的吞吐量记为相连边的边权值*/
for(k:=1 to n):
	for(i:=1 to n):
		for(j:=1 to n):
			D[i][j]:=min(D[i][j],D[i][k],D[k][j]) /*选取3者的最小值,作为最大吞吐量*/
\end{lstlisting}
\newpage
{\subsection*{problem 13.7}}
\markright{problem 13.7}
\addcontentsline{toc}{subsection}{problem 13.7}
计算出所有点到达$v_0$的最短路径和$v_0$到达其他顶点的最短路径。则$uv$间的最短路径即为
$u$到$v_0$的最短路径加上$v$到$v_0$的最短路径。
\begin{lstlisting}[language=C++,title=ShortByV.func]
ShortByV(G,v):
	D:=W /*初始化所有最短路径为起始结点权值*/
	for(k:=1 to n): /*计算所有节点到达v的最短路径*/
		for(i:=1 to n):
			D[i][v]:=min(D[i][v],D[i][k]+D[k][v])
	for(k:=1 to n): /*计算v到达所有节点的最短路径*/
		for(i:=1 to n):
			D[v][i]:=min(D[v][i],D[v][k]+D[k][i])
	for(i:=1 to n):
		for(j:=1 to n):
			D[i][j]:=D[i][v]+D[v][j] /*最短路径为i到v加v到j的最短路径*/
\end{lstlisting}
\section*{Chapter 14}
\markright{Chapter 14}
\addcontentsline{toc}{section}{Chapter 14}
\hypertarget{14.2}{\subsection*{problem 14.2}}
\markright{problem 14.2}
\addcontentsline{toc}{subsection}{problem 14.2}
\paragraph*{解}取子问题为$dp[i][j]$,含义为使用前i个数能否满足元素和为j。
对于$dp[i][j]$要么使用了第i个元素,要么使用了前i-1个元素。故其递推式为:
$$dp[i][j]=(dp[i-1][j])||(dp[i][j-A[i]]).$$
基于此得到整数子集和算法,\textbf{时间复杂度} $O(nS)$
,\textbf{空间复杂度} $O(nS)$:
\begin{lstlisting}[language=C++,title=SubSeqSum.func]
SubSeqSum(A,S):
	dp:=[n+1][S+1] /*dp数组大小为(n+1)(S+1)的二维数组*/
	for i:=1 to n do: /*初始化布尔dp数组*/
		dp[i][0]:=dp[i][A[i]]:=true
	for i:=2 to n do:
		for j:=1 to S do:
			if A[i]>j: /*如果A[i]已经大于j,必不能包含A[i]*/
				dp[i][j]:=dp[i-1][j]
			else:
				dp[i][j]:=(dp[i-1][j])||(dp[i][j-A[i]])
	return dp[n][s] /*返回使用n个数能否达到s即可*/
\end{lstlisting}

{\subsection*{problem 14.3}}
\markright{problem 14.3}
\addcontentsline{toc}{subsection}{problem 14.3}
\paragraph*{解}取子问题$dp[i]$为正整数i变为1需要的最少操作数。
根据上述规则,可根据子问题i与2,3的倍数关系,分成三种情况列出递推式:
$$\left\{
\begin{aligned}
	&i=3k\&\&i=2k,& dp[i]=min(d[i-1],d[i/2],d[i/3])+1,\\
	&i=2k\&\&i!=3k,& dp[i]=min(d[i-1],d[i/2])+1,\\
	&i=3k\&\&i!=2k,& dp[i]=min(d[i-1],d[i/3])+1,\\
	&i!=3k\&\&i!=2k,& dp[i]=d[i-1]+1.\\
\end{aligned}\right.
$$基于此得到算法伪代码,\textbf{时间复杂度} $O(n)$
,\textbf{空间复杂度} $O(n)$::
\begin{lstlisting}[language=C++,title=MiniOp.func]
MiniOp(n):
	dp:=[n+1] /*dp数组大小为n+1的一维数组*/
	dp[1]:=dp[2]:=dp[3]:=1 /*1,2,3均只需要1次操作*/
	for i:=4 to n do:
		if i%6=0:
			dp[i]:=min(d[i-1],d[i/2],d[i/3])+1
		else if i%3=0:
			dp[i]:=min(d[i-1],d[i/3])+1
		else if i%2=0:
			dp[i]:=min(d[i-1],d[i/2])+1
		else:
			dp[i]:=d[i-1]+1
	return dp[n] /*返回子问题规模为n的最少操作数即可*/
\end{lstlisting}
\newpage
{\subsection*{problem 14.6}}
\markright{problem 14.6}
\addcontentsline{toc}{subsection}{problem 14.6}
\paragraph*{1)}取子问题$dp[i][j]$记为X的前i个字符与Y的前j个字符的
LCS长度。
基于$X[i]$与$Y[j]$是否相等得到递推方程:
$$\left\{
	\begin{aligned}
		&X[i]==Y[j],&dp[i][j]=dp[i-1][j-1]+1,\\
		&X[i]!=Y[j],&dp[i][j]=max(dp[i-1][j],dp[i][j-1]).\\
	\end{aligned}
\right.
$$
基于此得到算法伪代码,\textbf{时间复杂度} $O(mn)$
,\textbf{空间复杂度} $O(mn)$:
\begin{lstlisting}[language=C++,title=LCS.func]
LCS(X,Y):
	dp:=[m+1][n+1] /*dp数组大小为(m+1)(n+1)的二维数组*/
	for i:=1 to m do:
		dp[i][0]:=0 /*X子串与空字符串的LCS为0*/
	for j:=1 to n do:
		dp[0][j]:=0 /*Y子串与空字符串的LCS为0*/
	for i:=1 to m do:
		for j:=1 to n do:
			if X[i]=Y[j]:
				dp[i][j]:=dp[i-1][j-1]+1
			else:
				dp[i][j]:=max(dp[i-1][j],dp[i][j-1])
	return dp[m][n] /*返回子问题规模为X和Y的LCS即可*/
\end{lstlisting}
\paragraph*{2)}
X中的字符可以重复出现,当$X[i]=Y[j]$时,对于子问题$dp[i][j]$,
其等价于X的前i个字符与Y前j-1个字符的LCS加1。
递推方程修改为:
$$\left\{
	\begin{aligned}
		&X[i]==Y[j],&dp[i][j]=dp[i][j-1]+1,\\
		&X[i]!=Y[j],&dp[i][j]=max(dp[i-1][j],dp[i][j-1]).\\
	\end{aligned}
\right.
$$
\begin{lstlisting}[language=C++,title=ReLCS.func]
ReLCS(X,Y):
	dp:=[m+1][n+1] /*dp数组大小为(m+1)(n+1)的二维数组*/
	for i:=1 to m do:
		dp[i][0]:=0 /*X子串与空字符串的LCS为0*/
	for j:=1 to n do:
		dp[0][j]:=0 /*Y子串与空字符串的LCS为0*/
	for i:=1 to m do:
		for j:=1 to n do:
			if X[i]=Y[j]:
				dp[i][j]:=dp[i][j-1]+1 /*递推式进行修改*/
			else:
				dp[i][j]:=max(dp[i-1][j],dp[i][j-1])
	return dp[m][n] /*返回子问题规模为X和Y的LCS即可*/
\end{lstlisting}
时间复杂度与空间复杂度不变。
\paragraph*{3)}
在循环中记录重复次数,当$X[i]=Y[j]$且$X[i]$重复次数超过k后,
$dp[i][j]$的LCS等价于X的前i-1个字符和Y前j-1个字符的LCS加1。
\begin{lstlisting}[language=C++,title=LimReLCS.func]
LimReLCS(X,Y,k):
	dp:=[m+1][n+1] /*dp数组大小为(m+1)(n+1)的二维数组*/
	for i:=1 to m do:
		dp[i][0]:=0 /*X子串与空字符串的LCS为0*/
	for j:=1 to n do:
		dp[0][j]:=0 /*Y子串与空字符串的LCS为0*/
	repnum:=0 /*重复次数初始化为0*/
	for i:=1 to m do:
		for j:=1 to n do:
			if X[i]=Y[j]:
				if repnum<k:
					repnum:=repnum+1
					dp[i][j]:=dp[i][j-1]+1 /*递推式进行修改*/
				else:
					dp[i][j]:=dp[i-1][j-1]+1
			else:
				repnum:=0 /*X[i]!=Y[j]即不再使用X[i]*/
				dp[i][j]:=max(dp[i-1][j],dp[i][j-1])
	return dp[m][n] /*返回子问题规模为X和Y的LCS即可*/
\end{lstlisting}
\newpage
\hypertarget{14.7}{\subsection*{problem 14.7}}
\markright{problem 14.7}
\addcontentsline{toc}{subsection}{problem 14.7}
\paragraph*{解}取子问题$dp[i][j]$记为以$T[i]$为结尾的前向串,
以$T[j]$为开始的后向串的最大连续子串长度。
为了满足两子串不重叠,易得$i<j$。
根据$T[i]$与$T[j]$得递推方程:
$$\left\{
	\begin{aligned}
		&T[i]==T[j]\&\&i<j,&dp[i][j]=dp[i-1][j+1]+1,\\
		&T[i]!=T[j]\&\&i<j,&dp[i][j]=0.\\
	\end{aligned}
\right.
$$
基于此得到算法伪代码,\textbf{时间复杂度} $O(n^2)$
,\textbf{空间复杂度} $O(n^2)$:
\begin{lstlisting}[language=C++,title=FroLas.func]
FroLas(T):
	dp:=[n+1][n+1] /*dp数组大小为(m+1)(n+1)的二维数组*/
	res:=-Inf /*初始化为最小值,结果保存最大的连续长度*/
 	for i:=1 to n do:
		dp[i][0]:=dp[0][i]:=0 /*子串与空字符串的重叠长度为0*/
	for i:=1 to n do:   /*前向串从头向后扩充*/
		for j:=n to i+1 do:  /*前向串从尾部向前扩充*/
			if T[i]=T[j]:
				dp[i][j]:=dp[i-1][j+1]+1
			else:
				dp[i][j]:=0
			res:=max(res,dp[i][j]) /*更新结果*/
	return res
\end{lstlisting}
{\subsection*{problem 14.11}}
\markright{problem 14.11}
\addcontentsline{toc}{subsection}{problem 14.11}
\paragraph*{1)}与 \hyperlink{14.7}{problem 14.7}类似,
取子问题$dp[i][j]$为子串$T[i\cdots j]$的最长回文子序列。
同样根据$T[i]$与$T[j]$得到递推方程:
$$\left\{
	\begin{aligned}
		&T[i]==T[j]\&\&i<j,&dp[i][j]=dp[i+1][j-1]+2,\\
		&T[i]!=T[j]\&\&i<j,&dp[i][j]=max(dp[i][j-1],dp[i+1][j]).\\
	\end{aligned}
\right.
$$
基于此得到算法伪代码,\textbf{时间复杂度} $O(n^2)$
,\textbf{空间复杂度} $O(n^2)$:
\begin{lstlisting}[language=C++,title=PalindromSeq.func]
PalindromSeq(T):
	dp:=[n+1][n+1] /*dp数组大小为(n+1)(n+1)的二维数组*/
 	for i:=1 to n do:
		dp[i][i]:=1 /*单个字符的回文串长度为1*/
	for i:=1 to n do:
		for j:=i to n-i-1 do:  /*枚举i-j的子串*/
			if T[i]=T[j]:
				dp[i][j]:=dp[i+1][j-1]+2
			else:
				dp[i][j]:=max(dp[i][j-1],dp[i+1][j])
	return dp[0][n-1] /*返回原字符串的最大回文序列长度*/
\end{lstlisting}
\paragraph*{2)}采用类似的算法,将字符串中的所有回文字串划分。
取子问题$IsPalind[i][j]$子串从i到j是否为回文子串,递推方程为:
$$
IsPalind[i][j]=(IsPalind[i+1][j-1])\&(T[i]==T[j]).
$$
取子问题$dp[i]$为划分前i个字符串需要的划分次数,如果$j-i$为回文子串,
那么$dp[i]\geq dp[j]+1$。因此
递推方程为:
$$
dp[i]=\min(dp[i],(IsPalind[j][i])\&\&(dp[j]+1)),j=1,2,\cdots, i-1.
$$
基于此得到算法伪代码,\textbf{时间复杂度} $O(n^2)$
,\textbf{空间复杂度} $O(n^2)$:
\begin{lstlisting}[language=C++,title=PalindSplit.func]
FindPalind(T):
	IsPalind[n+1][n+1] /*IsPalind数组大小为(n+1)(n+1)的二维数组*/
	for i:=1 to n do:
		IsPalind[i][i]:=true /*单个字符均为回文串*/
	for i:=1 to n do:
		for j:=1 to i do:  /*枚举j-i的子串*/
			IsPalind[i][j]:=(IsPalind[i+1][j-1])&(T[i]=T[j])

PalindSplit(T):
	FindPalind(T) /*将字符串中所有的回文子串进行标记*/
	dp:=[n+1] /*dp数组大小为(n+1)的一维数组*/
 	for i:=1 to n do:
		dp[i]:=i-1 /*初始化划分i个字符需要i-1次*/
	for i:=1 to n do:
		for j:=1 to i do:   /*枚举j-i的子串*/
			if IsPalind[i][j]:
				dp[i]:=min(dp[i],dp[j]+1)
	return dp[n-1]+1 /*返回原字符串最小划分次数+1,即回文串数量*/
\end{lstlisting}
{\subsection*{problem 14.13}}
\markright{problem 14.13}
\addcontentsline{toc}{subsection}{problem 14.13}
\paragraph*{1)}取子问题$dp[i]$记录是否可以表示金额i。
枚举使用每个硬币且不超过金额v的情况,如果$dp[j]$为真,
则$dp[x_k]$也为真,递推方程为:
$$dp[j+x_k]:=dp[j],k=1,2\cdots,n.$$
\begin{lstlisting}[language=C++,title=RepCoin.func]
RepCoin(X,v):
	dp:=[v+1] /*dp数组大小为(v+1)的一维数组*/
	for k:=1 to n do: /*所有硬币整数倍的均可以拼凑*/
		j:=0
		while(j*X[j]<=v):
			dp[j*X[j]]:=true
			j++
	for k:=1 to n do:
		for j:=0 to v do:
			if dp[j]=true:
				dp[j+X[i]]:=true
	return dp[v]
\end{lstlisting}
\paragraph*{2)}
与 \hyperlink{14.2}{problem 14.2} 类似,
取子问题为$dp[i][j]$,含义为使用前i个数能否满足金额和为j。
对于$dp[i][j]$要么使用了第i个元素,要么使用了前i-1个元素。故其递推式为:
$$dp[i][j]=(dp[i-1][j])||(dp[i][j-A[i]]).$$
\paragraph*{3)}即找最少的硬币来兑换目标金额。
取子问题$dp[i]$为组成金额$i$所需最少的硬币数量,则对应的递推方程为:
$$dp[i]=\min(dp(i-x_k))+1,k=0,1,\cdots,n.$$
\begin{lstlisting}[language=C++,title=MinRepCoin.func]
MinRepCoin(X,v):
	dp:=[v+1] /*dp数组大小为(v+1)的一维数组*/
	for i:=0 to v do: /*枚举0-v金额*/
		for j:=1 to n do: /*枚举使用每一枚硬币*/
			if X[j]<=i: /*硬币面值不超过所需金额*/
				dp[i]:=min(dp[i],dp[i-X[j]]+1)
	return dp[v]
\end{lstlisting}
{\subsection*{problem 14.14}}
\markright{problem 14.14}
\addcontentsline{toc}{subsection}{problem 14.14}
\paragraph*{解}
每个点设有两种状态:\\
$dp[i][0]$表示点i属于点覆盖集,并且以点i为根的子树中所连接的边都
被覆盖的情况下,点覆盖集中所包含最少点的个数。\\
$dp[i][1]$表示点i不属于点覆盖集,并且以点i为根的子树中所连接的边都
被覆盖的情况下,点覆盖集中所包含最少点的个数。\\
对于第一种状态,其最小个数等于所有子节点两种状态中最小值之和并加其本身。\\
对于第二种状态,因为点i所连接的边都应被覆盖,故点i的子节点都应被加入点覆盖集。
故其最小个数等于所有子节点的第一种状态之和。\\
$$\left\{
	\begin{aligned}
		&dp[i][0]=\sum\min(dp[j][0],dp[j][1]),parent[j]=i,\\
		&dp[i][1]=\sum(dp[j][0]),parent[j]=i.
	\end{aligned}
	\right.
$$
\begin{lstlisting}[language=C++,title=MiniVertexCover.func]
MiniVertexCover(u):
	dp[u][0]:=1;
	dp[u][1]:=0;
	for child v of u do:
         MiniVertexCover(v)
         dp[u][0]+=min(dp[v][0],dp[v][1])
		 dp[u][1]+=dp[v][0]

MiniVertexCoverWrapper(T,root):
	dp:=[V+1][2] /*dp数组大小为2(V+1)的一维数组*/
	dp(root)
	return min(dp[root][0],dp[root][1])
\end{lstlisting}
\end{document}