\documentclass[11pt,a4paper,oneside,oldfontcommands]{ctexart}
\usepackage{tabularx}
\usepackage{array}
\usepackage{bm}
\usepackage{hyperref}
\usepackage{graphicx}
\usepackage{amsmath}
\usepackage{algorithm}
\usepackage{algpseudocode}
\usepackage{fancyhdr}
\pagestyle{fancy}

\hypersetup{hypertex=true,
	colorlinks=true,
	linkcolor=red,
	anchorcolor=blue,
	citecolor=blue}
\fancyhf{}
\chead{\textbf{算法设计与分析}}
\fancyhead[r]{\bfseries\thepage}
\fancyhead[l]{\bfseries\rightmark}
\renewcommand{\headrulewidth}{0.4pt} % 注意不用 \setlength
\renewcommand{\footrulewidth}{0pt}

\floatname{algorithm}{算法}
\renewcommand{\algorithmicrequire}{\textbf{输入:}}
\renewcommand{\algorithmicensure}{\textbf{输出:}}

\begin{titlepage}
	\title{\Huge\textbf{算法设计与分析 作业五}\\}
	\author{\Large\textbf{作者}:吴润泽 \and{\Large\textbf{学号}:181860109}\\
	\\
	\and {\Large\textbf{Email}:\href{mailto:181860109@smail.nju.edu.cn}{181860109@smail.nju.edu.cn}}\\}
	\date{\Large\today}
\end{titlepage}
\begin{document}
\maketitle
\newpage
\tableofcontents
\cleardoublepage
\section*{Chapter 10}
\markright{Chapter 10}
\addcontentsline{toc}{section}{Chapter 10}
{\subsection*{problem 10.3}}
\markright{problem 10.3}
\addcontentsline{toc}{subsection}{problem 10.3}
{\subsection*{problem 10.4}}
\markright{problem 10.4}
\addcontentsline{toc}{subsection}{problem 10.4}
{\subsection*{problem 10.8}}
\markright{problem 10.8}
\addcontentsline{toc}{subsection}{problem 10.8}
{\subsection*{problem 10.9}}
\markright{problem 10.9}
\addcontentsline{toc}{subsection}{problem 10.9}
{\subsection*{problem 10.11}}
\markright{problem 10.11}
\addcontentsline{toc}{subsection}{problem 10.11}
{\subsection*{problem 10.13}}
\markright{problem 10.13}
\addcontentsline{toc}{subsection}{problem 10.13}
{\subsection*{problem 10.15}}
\markright{problem 10.15}
\addcontentsline{toc}{subsection}{problem 10.15}
{\subsection*{problem 10.16}}
\markright{problem 10.16}
\addcontentsline{toc}{subsection}{problem 10.16}
{\subsection*{problem 10.18}}
\markright{problem 10.18}
\addcontentsline{toc}{subsection}{problem 10.18}
{\subsection*{problem 10.19}}
\markright{problem 10.19}
\addcontentsline{toc}{subsection}{problem 10.19}
{\subsection*{problem 10.21}}
\markright{problem 10.21}
\addcontentsline{toc}{subsection}{problem 10.21}
\newpage
\section*{Chapter 11}
\markright{Chapter 11}
\addcontentsline{toc}{section}{Chapter 11}
{\subsection*{problem 11.1}}
\markright{problem 11.1}
\addcontentsline{toc}{subsection}{problem 11.1}
{\subsection*{problem 11.2}}
\markright{problem 11.2}
\addcontentsline{toc}{subsection}{problem 11.2}
{\subsection*{problem 11.6}}
\markright{problem 11.6}
\addcontentsline{toc}{subsection}{problem 11.6}
{\subsection*{problem 11.8}}
\markright{problem 11.8}
\addcontentsline{toc}{subsection}{problem 11.8}
{\subsection*{problem 11.9}}
\markright{problem 11.9}
\addcontentsline{toc}{subsection}{problem 11.9}
{\subsection*{problem 11.10}}
\markright{problem 11.10}
\addcontentsline{toc}{subsection}{problem 11.10}
{\subsection*{problem 11.12}}
\markright{problem 11.12}
\addcontentsline{toc}{subsection}{problem 11.12}
\newpage
\section*{Chapter 13}
\markright{Chapter 13}
\addcontentsline{toc}{section}{Chapter 13}
{\subsection*{problem 13.1}}
\markright{problem 13.1}
\addcontentsline{toc}{subsection}{problem 13.1}
{\subsection*{problem 13.2}}
\markright{problem 13.2}
\addcontentsline{toc}{subsection}{problem 13.2}
{\subsection*{problem 13.5}}
\markright{problem 13.5}
\addcontentsline{toc}{subsection}{problem 13.5}
{\subsection*{problem 13.6}}
\markright{problem 13.6}
\addcontentsline{toc}{subsection}{problem 13.6}
{\subsection*{problem 13.7}}
\markright{problem 13.7}
\addcontentsline{toc}{subsection}{problem 13.7}
{\subsection*{problem 13.9}}
\markright{problem 13.9}
\addcontentsline{toc}{subsection}{problem 13.9}
\end{document}