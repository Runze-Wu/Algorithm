\documentclass[11pt,a4paper,oneside,oldfontcommands]{ctexart}
\usepackage{tabularx}
\usepackage{array}
\usepackage{bm}
\usepackage{hyperref}
\usepackage{graphicx} 
\usepackage{amsmath}
\usepackage{algorithm}
\usepackage{algpseudocode}
\usepackage{fancyhdr}
\pagestyle{fancy}
\usepackage{color}
\usepackage{xcolor}
\definecolor{keywordcolor}{rgb}{0.8,0.1,0.5}
\usepackage{listings}


\definecolor{mygreen}{rgb}{0,0.6,0}  
\definecolor{mygray}{rgb}{0.5,0.5,0.5}  
\definecolor{mymauve}{rgb}{0.58,0,0.82}  
  
\lstset{ %  
  backgroundcolor=\color{white},   % choose the background color; you must add \usepackage{color} or \usepackage{xcolor}  
  basicstyle=\footnotesize,        % the size of the fonts that are used for the code  
  breakatwhitespace=false,         % sets if automatic breaks should only happen at whitespace  
  breaklines=true,                 % sets automatic line breaking  
  captionpos=bl,                    % sets the caption-position to bottom  
  commentstyle=\color{mygreen},    % comment style  
  deletekeywords={...},            % if you want to delete keywords from the given language  
  escapeinside={\%*}{*)},          % if you want to add LaTeX within your code  
  extendedchars=true,              % lets you use non-ASCII characters; for 8-bits encodings only, does not work with UTF-8  
  frame=single,                    % adds a frame around the code  
  keepspaces=true,                 % keeps spaces in text, useful for keeping indentation of code (possibly needs columns=flexible)  
  keywordstyle=\color{blue},       % keyword style  
  %language=Python,                 % the language of the code  
  morekeywords={*,...},            % if you want to add more keywords to the set  
  numbers=left,                    % where to put the line-numbers; possible values are (none, left, right)  
  numbersep=5pt,                   % how far the line-numbers are from the code  
  numberstyle=\tiny\color{mygray}, % the style that is used for the line-numbers  
  rulecolor=\color{black},         % if not set, the frame-color may be changed on line-breaks within not-black text (e.g. comments (green here))  
  showspaces=false,                % show spaces everywhere adding particular underscores; it overrides 'showstringspaces'  
  showstringspaces=false,          % underline spaces within strings only  
  showtabs=false,                  % show tabs within strings adding particular underscores  
  stepnumber=1,                    % the step between two line-numbers. If it's 1, each line will be numbered  
  stringstyle=\color{orange},     % string literal style  
  tabsize=3,                       % sets default tabsize to 2 spaces  
  %title=myPython.py                   % show the filename of files included with \lstinputlisting; also try caption instead of title  
}  
\hypersetup{hypertex=true,
	colorlinks=true,
	linkcolor=red,
	anchorcolor=blue,
	citecolor=blue}
\fancyhf{}
\chead{\textbf{算法设计与分析}}
\fancyhead[r]{\bfseries\thepage}
\fancyhead[l]{\bfseries\rightmark}
\renewcommand{\headrulewidth}{0.4pt} % 注意不用 \setlength
\renewcommand{\footrulewidth}{0pt}

\floatname{algorithm}{算法}
\renewcommand{\algorithmicrequire}{\textbf{输入:}}
\renewcommand{\algorithmicensure}{\textbf{输出:}}

\begin{titlepage}
	\title{\Huge\textbf{算法设计与分析 作业七}\\}
	\author{\Large\textbf{作者}:吴润泽 \and{\Large\textbf{学号}:181860109}\\
	\\
	\and {\Large\textbf{Email}:\href{mailto:181860109@smail.nju.edu.cn}{181860109@smail.nju.edu.cn}}\\}
	\date{\Large\today}
\end{titlepage}
\begin{document}
\maketitle

\tableofcontents
\cleardoublepage
\section*{Chapter 15}
\markright{Chapter 15}
\addcontentsline{toc}{section}{Chapter 15}
{\subsection*{problem 15.1}}
\markright{problem 15.1}
\addcontentsline{toc}{subsection}{problem 15.1}
\subsubsection*{(1)}
\paragraph*{CLIQUE}
\subparagraph*{优化问题}
\begin{itemize}
	\item {输入实例: 无向图G}
	\item {优化问题:求图G中最大团的大小}
\end{itemize}
\subparagraph*{判定问题}
\begin{itemize}
	\item {输入实例: 无向图G;参数k}
	\item {判定问题:图G中是否存在大小为k的团}
\end{itemize}
\paragraph*{KNAPSACK}
\subparagraph*{优化问题}
\begin{itemize}
	\item {输入实例: n个物品,其大小分别为$s_1,s_2,\cdots,s_n$,每个物品的价值为$c_1,c_2,\cdots,c_n$;参数C}
	\item {优化问题:背包中装若干物品,使得背包中物品的大小之和不超过C的物品价值和的最大值}
\end{itemize}
\subparagraph*{判定问题}
\begin{itemize}
	\item {输入实例: n个物品,其大小分别为$s_1,s_2,\cdots,s_n$,每个物品的价值为$c_1,c_2,\cdots,c_n$;参数k和C}
	\item {判定问题:是否可以在背包中装若干物品,使得背包中物品的大小之和不超过C,且价值之和不低于k}
\end{itemize}
\paragraph*{INDEPENDENT-SET}
\subparagraph*{优化问题}
\begin{itemize}
	\item {输入实例: 无向图G}
	\item {优化问题:求图G中最大独立集的大小}
\end{itemize}
\subparagraph*{判定问题}
\begin{itemize}
	\item {输入实例: 无向图G;参数k}
	\item {判定问题:图G中是否存在大小为k的独立集}
\end{itemize}
\paragraph*{VERTEX-COVER}
\subparagraph*{优化问题}
\begin{itemize}
	\item {输入实例: 无向图G}
	\item {优化问题:求图G中最小点覆盖的大小}
\end{itemize}
\subparagraph*{判定问题}
\begin{itemize}
	\item {输入实例: 无向图G;参数k}
	\item {判定问题:图G中是否存在大小为k的点覆盖}
\end{itemize}
\subsubsection*{(2)}
\paragraph*{CLIQUE}
\subparagraph*{优化问题多项式时间可解} 设其最大团大小为n,对于其判定问题,
输入参数为k。如果$k\leq n$,则判定结果为$true$,否则结果为$false$。因此判定问题也是多项式时间可解。
\subparagraph*{判定问题多项式时间可解} 假设图G点数为n,则分别取判定问题输入参数$k=1,2,\cdots,n$,
当$k<m$时结果均为$true$,$k=m$时结果为$false$,则图G的最大团大小即为$m-1$,则最多进行了n次的判定问题求解。
由于多项式的计算封闭性,优化问题也为多项式时间可解。
\paragraph*{KNAPSACK}
\subparagraph*{优化问题多项式时间可解} 设对于给定的若干物品,以及容量$C^{'}$,其最大的物品价值和为m,
对于其判定问题,令输入参数$C=C^{'}$,其另一输入参数k。如果$k\leq m$,则判定结果为$true$,否则结果为$false$。因此判定问题也是多项式时间可解。
\subparagraph*{判定问题多项式时间可解} 对于给定容量$C$,分别取判定问题输入参数$k=sum(1),sum(2),\cdots,sum(n)$,
$sum(i)$为前i个物品价值总和,
当$k<m$时结果均为$true$,$k=m$时结果为$false$,则背包中物品的大小之和不超过C的物品价值和的最大值即为$m-1$,
则最多进行了n次的判定问题求解。
由于多项式的计算封闭性,优化问题也为多项式时间可解。
\paragraph*{INDEPENDENT-SET}
\subparagraph*{优化问题多项式时间可解}设其最大独立集大小为n,对于其判定问题,
输入参数为k。如果$k\leq n$,则判定结果为$true$,否则结果为$false$。因此判定问题也是多项式时间可解。
\subparagraph*{判定问题多项式时间可解}假设图G点数为n,则分别取判定问题输入参数$k=1,2,\cdots,n$,
当$k<m$时结果均为$true$,$k=m$时结果为$false$,则图G的最大独立集大小即为$m-1$,则最多进行了n次的判定问题求解。
由于多项式的计算封闭性,优化问题也为多项式时间可解。
\paragraph*{VERTEX-COVER}
\subparagraph*{优化问题多项式时间可解}设其最小点覆盖大小为n,对于其判定问题,
输入参数为k。如果$k\geq n$,则判定结果为$true$,否则结果为$false$。因此判定问题也是多项式时间可解。
\subparagraph*{判定问题多项式时间可解}假设图G点数为n,则分别取判定问题输入参数$k=n,n-1,\cdots,1$,
当$k\geq m$时结果均为$true$,$k<m$时结果为$false$,则图G的最小点覆盖大小即为$m$,则最多进行了n次的判定问题求解。
由于多项式的计算封闭性,优化问题也为多项式时间可解。
{\subsection*{problem 15.2}}
\markright{problem 15.2}
\addcontentsline{toc}{subsection}{problem 15.2}
\paragraph*{证明}即证明如果有$L_1\leq_pL_2$,且$L_2\leq_pL_3$则有$L_1\leq_pL_3$。
因为$L_1\leq_pL_2$,则问题$L_1$可以通过多项式时间转换函数$T_1$归约到问题$L_2$,同理,
问题$L_2$可以通过多项式时间转换函数$T_2$归约到问题$L_3$。由多项式计算封闭性可知,$L_1$可以通过
多项式时间转换$T_3$归约到问题$L_3$,因此$L_1\leq_pL_3$。
{\subsection*{problem 15.4}}
\markright{problem 15.4}
\addcontentsline{toc}{subsection}{problem 15.4}
设原始序列为$a_1,a_2,\cdots,a_n$,选择输入参数k,代表选择第k大。
\paragraph*{排序归约到选择}
对于排序的原始序列直接设为传入选择算法的序列即可,输入转换代价为常数,
对序列进行n次选择,输入参数$k=1,2,\cdots,n$,即依次选择第i大元素将其放在目标序列的对应位置。
n次选择后,即可得到已经排好序的序列,将其输出即可。
\paragraph*{选择归约到排序}对于选择的原始序列直接设为传入排序算法的序列即可,输入转换代价为常数,
然后根据选择算法的输入参数k,获取排序后序列的下标为k的元素作为输出即可。

\noindent 与选择排序相似,每次选择待排序序列的最大元素加入已排序的末尾。
{\subsection*{problem 15.5}}
\markright{problem 15.5}
\addcontentsline{toc}{subsection}{problem 15.5}
\paragraph*{问题1归约到问题2}
对于问题1的输入集合S直接设为传入问题2的输入集合即可,并令问题2的输入参数$k=\frac{n+1}{2}$,
输出即为集合S的中位数。
\paragraph*{问题2归约到问题1}不妨设n为奇数,阶为k的元素为m\\
	1. 当k等于$\frac{n+1}{2}$时,将原集合传入问题1,即可找到m;\\
	2. 当k小于$\frac{n+1}{2}$时,遍历原集合S,记录其元素最小值为min,对于原集合S,有$n-k$个元素大于m,$k-1$个元素小于m。\\
	3. 开辟新集合temp,将前$n-2k+1$个元素值赋为min-1,并将S的n个元素放入其后。
	则集合temp中有$n-k$个元素大于m,$n-k$个元素小于m。因此m为temp中位数,将temp传入问题1,即可找到m。\\
	4. 当k大于$\frac{n+1}{2}$时,同理记录其元素最大值为max,开辟数组temp,在数组temp中前$2k-n-1$个赋值为max+1。同样使得m变为temp中位数,将temp传入问题1,即可找到m。\\
\newpage
\section*{Chapter 16}
\markright{Chapter 16}
\addcontentsline{toc}{section}{Chapter 16}
{\subsection*{problem 16.2}}
\markright{problem 16.2}
\addcontentsline{toc}{subsection}{problem 16.2}
\subsubsection*{1)}
对于判定算法,选择图G中的k个点,共有$C_n^k=\frac{n!}{(n-k)!k!}$种选择,
选择k个点后,判断k个点之间是否均有边相连,则最多遍历$C_k^2=\frac{k(k-1)}{2}$条边。
算法最多需要判断所有可能选择是否成立,因此时间复杂度为$T(n)=\frac{n!}{(n-k)!k!}\times\frac{k(k-1)}{2}$,
同时k为常数,则$T(n)=O(n^k)$,即为多项式复杂度。
\begin{lstlisting}[language=C++,title=FakeCliqueJudge.func]
FakeCliqueJudge(G,k)
Loop:
	for each G_k in G:  /*选取G中的每个有k个点的子图*/
		for v in G_k:
			for w in G_k:
				if v=w: continue;
				if vw not in G: goto Loop;
				/*如果图G中不存在vw边,则判断下一个k点子图*/
		return true; /*每一对点均有边,则说明为团*/
	return false;
\end{lstlisting}
\subsubsection*{2)}
不能证明。因为题中要求的判定问题中,是根据给定的常数k,判断是否具有k大小的团。
而$CLIQUE$问题的判定问题中,输入参数k是不确定的变量,在确定判定问题复杂度时,k为常数的条件不再成立。
伪最大团问题与$CLIQUE$问题并不等价,因此并不能证明$P=NP$。
\newpage
{\subsection*{problem 16.3}}
\markright{problem 16.3}
\addcontentsline{toc}{subsection}{problem 16.3}
\subsubsection*{1)}
对于DNF-SAT,每个子句间以逻辑或相连,因此若使整个表达式成立,则使得其中任意一个子句成立即可。
对于任意子句,其中各变量以逻辑与相连,只要其中不同时存在某一变量和该变量的非,
则就可以对子句中每一布尔变量进行相应的赋值,使其为$true$即可,就能使得子句为$true$。
如果每一子句都包含了布尔变量和布尔变量的非,则DNF-SAT中每一子句均不能为$true$,最终结果为$false$。
显然在$O(n)$时间内即可完成,即多项时间内可解,DNF-SAT为P问题。
\subsubsection*{2)}
CNF-SAT等价地转换为DNF-SAT时,这个归约过程无法保证在多项式时间内完成,所以后续推理不成立。
则不能得到$CNF-SAT\leq_p DNF-SAT$,不能证明$P=NP$。
{\subsection*{problem 16.4}}
\markright{problem 16.4}
\addcontentsline{toc}{subsection}{problem 16.4}
\paragraph*{稠密子图为NP问题}
对于任意猜测的解,(k个点的形式)
我们可以在$O(k^2)$的时间内验证,
这个子图中的边数是否至少有y条边,
即解可以多项式时间内验证,所以是NP问题。
\paragraph*{稠密子图为NP难问题}
可通过证明最大团问题可以在多项式时间下归约到稠密子图问题,来进行证明。
对于最大团问题其输入G,k直接作为稠密子图的输入G、k,
并令$y=C_k^2=\frac{k(k-1)}{2}$,显然归约为多项式时间。\\
\hspace*{20pt}\textbf{输出的正确性: }
对于一个子图H,有k个顶点,则其最多有$C_k^2$条边,即完全图的情况。
如果对于上述归约后的输入,稠密子图输出为$true$,则说明G中存在
k大小的完全图(团);稠密子图输出为$false$,则说明G中任意k个顶点的子图,
其边数均不足$C_k^2$,即不存在k大小的完全图。
\paragraph*{稠密子图为NP完成问题}
综上最大团问题可以多项式时间归约到稠密子图问题,
因为最大团问题是NP完全问题,
并且稠密子图是NP问题 ,
所以稠密子图是NP完全问题。

{\subsection*{problem 16.5}}
\markright{problem 16.5}
\addcontentsline{toc}{subsection}{problem 16.5}
\paragraph*{SET-COVER为NP问题}
给定k个子集,我们可以在O(kn)时间内验证它们的并集是否是U,
所以我们可以在多项式时间内,验证这个问题的一个解,所以这个问题是NP的。
\paragraph*{SET-COVER为NP难问题}
可通过证明DOMINATION-SET问题可以在多项式时间下归约到SET-COVER问题,来证明。
对于DOMINATION-SET问题其输入G,k,
以k作为SET-COVER的输入k,以图G的顶点集合作为输入U,每个子集$S_i$为图G第i个点以及该点的邻居。
显然归约为多项式时间。\\
\hspace*{20pt}\textbf{输出的正确性: }
如果对于上述归约后的输入,输出为$true$,则说明存在大小为k的集合$S_1^{'},S_2^{'},\cdots,S_k^{'}$覆盖U,
而$S_i^{'}$则对应了图G中的点$v_i^{'}$,即存在大小为k的支配集;如果输出为$false$,同理,
说明不存在大小为k的支配集。
\paragraph*{SET-COVER为NP完成问题}
综上DOMINATION-SET问题可以多项式时间归约到SET-COVER问题,
因为DOMINATION-SET问题是NP完全问题,
并且SET-COVER是NP问题 ,
所以稠SET-COVER问题是NP完全问题。
\end{document}