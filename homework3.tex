\documentclass[11pt]{ctexart}
\usepackage{tabularx}
\usepackage{array}
\usepackage{bm}
\usepackage{hyperref}
\usepackage{amsmath}
\usepackage{algorithm}
%\usepackage{algorithm2e}
\usepackage{algpseudocode}
\usepackage{fancyhdr}
\pagestyle{fancy}

\hypersetup{hypertex=true,
	colorlinks=true,
	linkcolor=red,
	anchorcolor=blue,
	citecolor=blue}
\fancyhf{}
\chead{\textbf{算法设计与分析}}
\fancyhead[r]{\bfseries\thepage}
\fancyhead[l]{\bfseries\rightmark}
\renewcommand{\headrulewidth}{0.4pt} % 注意不用 \setlength
\renewcommand{\footrulewidth}{0pt}

\floatname{algorithm}{算法}
\renewcommand{\algorithmicrequire}{\textbf{输入:}}
\renewcommand{\algorithmicensure}{\textbf{输出:}}

\begin{titlepage}
	\title{\Huge\textbf{算法设计与分析 作业三}\\}
	\author{\Large\textbf{作者}:吴润泽 \and{\Large\textbf{学号}:181860109}\\
	\\
	\and {\Large\textbf{Email}:\href{mailto:181860109@smail.nju.edu.cn}{181860109@smail.nju.edu.cn}}\\}
	\date{\Large\today}
\end{titlepage}

\begin{document}
	\maketitle
	\tableofcontents
	\newpage
	\section*{CHAPTER 8}
	\markright{CHAPTER 8}
	\addcontentsline{toc}{section}{CHAPTER 8}
	\hypertarget{problem 8.2}{\subsection*{problem 8.2}}
	\markright{problem 8.2}
	\addcontentsline{toc}{subsection}{problem 8.2}
	\paragraph{算法}
	假设有5个数a,b,c,d,e\\
	1.比较a b,将较小者放入a,较大者放入b\\
	2.比较c d,将较小者放入c,较大者放入d\\
	3.比较a c,将较小者放入a,较大者放入c,若a和c发生交换则同时交换b和d,即保证a b和
	c d原有大小关系不变。此时有$a<c<d,a<b$,则a不可能是中位数\\
	4.比较b e,将较小者放入b,较大者放入e\\
	5.比较b c,将较小者放入b,较大者放入c,若b和c发生交换则同时交换d和e,即保证b e和
	c d原有大小关系不变。此时有$b<c<d,b<e$,则b不可能是中位数\\
	6.比较c e,\hspace{5pt}将较小者放入c,较大者放入e,此时有$c<d<e$,且$c>b,c>a$,即c就是中位数。
	\newpage
	\hypertarget{problem 8.4}{\subsection*{problem 8.4}}
	\markright{problem 8.4}
	\addcontentsline{toc}{subsection}{problem 8.4}
	\paragraph{算法}
	设数组array元素个数为n,不妨设n为奇数,阶为k的元素为m;\\
	1. 当k等于$\frac{n+1}{2}$时,直接调用算法A,即可找到m;\\
	2. 当k小于$\frac{n+1}{2}$时,遍历数组array,记录其元素最小值为min,对于原数组array,有$n-k$个元素大于m,$k-1$个元素小于m。\\
	3. 开辟新数组temp,将前$n-2k+1$个元素值赋为min-1,并将array的n个元素放入其后。
	则数组temp中有$n-k$个元素大于m,$n-k$个元素小于m。因此m为temp中位数,调用算法A,即可找到m。\\
	4. 当k大于$\frac{n+1}{2}$时,同理记录其元素最大值为max,开辟数组temp,在数组temp中前$2k-n-1$个赋值为max+1。同样使得m变为temp中位数,调用算法A,即可找到m。\\
	\label{Kth-find算法}
	$\begin{aligned}
	\hline
	&\textbf{算法1 }KthFind\text{ 算法}\\
	\hline
	&1.\textbf{Function}\ {KthFind}\ (array,n,k)\\
	&2.\hspace*{20pt}\textbf{if }k==\frac{n+1}{2}\textbf{ return }A(array,n)\\
	&3.\hspace*{20pt}min\_num:= min(array,n),max\_num:= max(array,n)\\
	&4.\hspace*{20pt}Let\ temB[1...2n]\ be\ new\ array.\\
	&5.\hspace*{20pt}\textbf{for}\ i:= 1\ to\ n\ \textbf{do}\\
	&6.\hspace*{40pt}temB[i]:=A[i]\\
	&7.\hspace*{20pt}\textbf{if }k<\frac{n+1}{2} \textbf{ then}\\
	&8.\hspace*{40pt}\textbf{for}\ i:= 1\ to\ n-2k+1\ \textbf{do}\\
	&9.\hspace*{60pt}temB[i+n]:= min\_num-1\\
	&10.\hspace*{35pt}\textbf{return }A(temB,2n-2k+1)\\
	&11.\hspace*{15pt}\textbf{if }k>\frac{n+1}{2} \textbf{ then}\\
	&12.\hspace*{35pt}\textbf{for}\ i:= 1\ to\ 2k-n-1\ \textbf{do}\\
	&13.\hspace*{55pt}temB[i+n]:= max\_num-1\\
	&14.\hspace*{35pt}\textbf{return }A(temB,2k-1)\\
	\hline
	\end{aligned}
	$

	\paragraph{时间复杂度}寻找最值和开辟新数组的时间复杂度为$O(n)$,添加的元素个数最多为$n-1$个,数组的规模仍为$O(n)$,而找中位数算法A,时间复杂度为$O(n)$,因此总的时间复杂度仍为线性。
	\newpage
	\subsection*{problem 8.5}
	\markright{problem 8.5}
	\addcontentsline{toc}{subsection}{problem 8.5}
	\subsubsection*{(1)}使用归并排序进行降序排列,排序结果前k个元素即为所求。\\
	归并排序时间复杂度为$O(n\log n)$,输出复杂度为$O(k)$,总时间复杂度为$O(n\log n)$.\\
	$\begin{aligned}
	\hline
	&1.\textbf{Function}\ {KthOrder1}\ (array,n,k)\\
	&2.\hspace*{20pt}\text{对原数组使用归并排序,并按照升序排列}\\
	&3.\hspace*{20pt}\textbf{for}\ i:= 1\ to\ k\ \textbf{do}\\
	&4.\hspace*{40pt}res.add(array[i])\\
	&5.\hspace*{20pt}\textbf{return }res\\
	\hline
	\end{aligned}
	$
	\subsubsection*{(2)}根据原数组建立最大堆,最大堆堆顶存储当前堆的最大值,则弹出堆顶元素k次,k个元素即为所求。\\
	根据已有序列建堆的时间复杂度为$O(n)$,弹出堆顶元素k次,每次修复堆时间复杂度为$O(\log n)$,则总的时间复杂度为$O(n+k\log n)$.\\
	$\begin{aligned}
	\hline
	&1.\textbf{Function}\ {KthOrder2}\ (array,n,k)\\
	&2.\hspace*{20pt}\text{根据原数组进行建堆,得到最大堆为heap}\\
	&3.\hspace*{20pt}\textbf{for}\ i:= 1\ to\ k\ \textbf{do}\\
	&4.\hspace*{40pt}res.add(heap.top())\\
	&5.\hspace*{40pt}heap.pop()\verb|\\|\text{弹出堆顶元素,并修复}\\
	&6.\hspace*{20pt}\textbf{return }res\\
	\hline
	\end{aligned}
	$
	\hypertarget{problem 8.5(3)}{\subsubsection*{(3)}}
	利用 \hyperlink{problem 8.4}{problem 8.4} 中 \hyperref[Kth-find算法]{Kth-find算法} ,找到原数组中的第k+1大元素m。遍历原数组,找到其中大于m的元素加入结果数组res。 对res使用归并排序,升序排列,得到res即为所求。\\
	找第k+1大,遍历原数组为线性时间$O(n)$,对res归并排序$O(k\log k)$,总时间复杂度为$O(n+k\log k)$.\\
	$\begin{aligned}
	\hline
	&1.\textbf{Function}\ {KthOrder3}\ (array,n,k)\\
	&2.\hspace*{20pt}m:=KthFind(array,n,k+1)\verb|\\找到数组第k+1大元素|\\
	&3.\hspace*{20pt}\textbf{for}\ i:= 1\ to\ n\ \textbf{do}\\
	&4.\hspace*{40pt}\textbf{if }array[i]>m \textbf{ do } res.add(array[i])\\
	&5.\hspace*{20pt}\text{对res数组使用归并排序,并按照升序排列}\\
	&6.\hspace*{20pt}\textbf{return }res\\
	\hline
	\end{aligned}
	$
	\newpage
	\subsection*{problem 8.6}
	\markright{problem 8.6}
	\addcontentsline{toc}{subsection}{problem 8.6}
	\subsubsection*{(1)}使用归并排序进行排列时间复杂度为$O(n\log n)$,遍历中位数M左右两侧元素,与M差值最小的加入res,并移动对应侧指针,直至res个数为k即可。总时间复杂度为$O(n\log n+k)$.\\
	$\begin{aligned}
	\hline
	&1.\textbf{Function}\ {KthNear1}\ (array,n,k)\\
	&2.\hspace*{20pt}\text{对原数组使用归并排序}\\
	&3.\hspace*{20pt}l:=\frac{n+1}{2}-1,\ r:=\frac{n+1}{2}+1\\
	&4.\hspace*{20pt}\textbf{for}\ i:= 1\ to\ k\ \textbf{do}\\
	&5.\hspace*{40pt}\textbf{if }array[l]+array[r]>2M \textbf{ do } res.add(array[r++])\\
	&6.\hspace*{40pt}\textbf{else }res.add(array[l--])\\
	&7.\hspace*{20pt}\textbf{return }res\\
	\hline
	\end{aligned}
	$
	\subsubsection*{(2)}利用 \hyperlink{problem 8.4}{problem8.4} 中的查找中位数算法,得到中位数为M,时间复杂度为$O(n)$。
	\hspace*{20pt}将原数组分为大于M和小于M的两数组L,S,时间复杂度为$O(n)$。
	\hspace*{20pt}与 \hyperlink{problem 8.5(3)}{problem 8.5(3)} 同样思想,找到L的前k小(查找第k+1小,即第n-k-1大)LK和S的前k大元素SK,时间复杂度为$O(k)$。对LK进行归并排序按照升序排列,对SK进行归并排序按照降序排列,时间复杂度为$O(k\log k)$。
	之后遍历LK和SK找到与M最接近的K个元素即可。\\
	$\begin{aligned}
	\hline
	&1.\textbf{Function}\ {KthNear2}\ (array,n,k)\\
	&2.\hspace*{20pt}M:=A(array,n)\verb|\\|\text{找到中位数M}\\
	&3.\hspace*{20pt}\text{划分原数组为大于M和小于M两部分:L[1..n1],S[1..n2]}\\
	&4.\hspace*{20pt}large:=KthFind(L,n1,n1-k-1)\verb|\\找到L数组第k+1小元素|\\
	&5.\hspace*{20pt}small:=KthFind(S,n2,k+1)\verb|\\找到L数组第k+1大元素|\\
	&6.\hspace*{20pt}\text{找到L数组小于large的k个元素,并升序排列,得到LK}\\
	&7.\hspace*{20pt}\text{找到S数组大于small的k个元素,并降序排列,得到SK}\\
	&8.\hspace*{20pt}l:=0,\ r:=0\\
	&9.\hspace*{20pt}\textbf{for}\ i:= 1\ to\ k\ \textbf{do}\\
	&10.\hspace*{35pt}\textbf{if }LK[l]+SK[r]>2M \textbf{ do } res.add(SK[r++])\\
	&11.\hspace*{15pt}\textbf{else }res.add(LK[l--])\\
	&12.\hspace*{15pt}\textbf{return }res\\
	\hline
	\end{aligned}
	$
	\subsection*{problem 8.8}
	\markright{problem 8.8}
	\addcontentsline{toc}{subsection}{problem 8.8}
	\paragraph{算法设计}
	 使用两个堆,大根堆q1维护较小值,小根堆维护较大值,令大根堆q2元素个数为$m$,小根堆元素个数为$n$:\\
	 \hspace*{20pt}使得小根堆的堆顶是较大数中最小的,大根堆的堆顶是较小数中最大的;\\
	 \hspace*{20pt}将大于大根堆堆顶的数放小根堆,小于等于大根堆堆顶的数放大根堆;\\
	 对于大根堆的堆顶元素,有$n$个元素比该元素大,$m-1$个元素比该元素小;\\
	 对于小根堆的堆顶元素,有$m$个元素比该元素小, $n-1$个元素比该元素大;\\
	在维护$|m-n| \leq 1$之后,当$m=n$时,两堆顶元素均为中位数,当$m\ne n$时,元素个数较多的堆顶元素即为当前中位数;\\
	易知插入和删除的时间复杂度均为$O(\log n)$,查找中值为常数。\\
	$\begin{aligned}
	&\textbf{查找中值}\\
	\hline
	&1.\hspace*{20pt}\textbf{if }(q1.size()+q2.size())\%2==1\textbf{ then }\\
	&2.\hspace*{40pt}\textbf{if }q1.size()>q2.size()\textbf{ do }mid:=q1.top()\\
	&3.\hspace*{40pt}\textbf{else } mid:=q2.top()\\
	&4.\hspace*{20pt}\textbf{else } mid:=(q1.top+q2.top())/2\\
	\hline
	\end{aligned}\\
	$$
	\begin{aligned}
	&\textbf{插入操作}\\
	\hline
	&1.\hspace*{20pt}\textbf{if }input > q1.top()\textbf{ do }q2.push(input)\\
	&2.\hspace*{20pt}\textbf{else }myq1.push(input)\verb|\\|\text{大根堆放较小数,小根堆放较大数}\\
	&3.\hspace*{20pt}\textbf{while }|q1.size()-q2.size()|>1\textbf{ do}\\
	&4.\hspace*{40pt}\textbf{if }q1.size()>q2.size()\textbf{ do }q2.push(q1.pop())\\
	&4.\hspace*{40pt}\textbf{else } q1.push(q2.pop())\\
	\hline
	\end{aligned}\\
	$
	$
	\begin{aligned}
	&\textbf{删除操作}\\
	\hline
	&1.\hspace*{20pt}\textbf{if }(q1.size()+q2.size())\%2==1\textbf{ then }\\
	&2.\hspace*{40pt}\textbf{if }q1.size()>q2.size()\textbf{ do }q1.pop()\\
	&3.\hspace*{40pt}\textbf{else }q2.pop()\\
	&4.\hspace*{20pt}\textbf{else }q2.pop()\verb|\\|\text{当为偶数时任意弹出一个}\\
	\hline
	\end{aligned}
	$
	\subsection*{problem 8.9}
	\markright{problem 8.9}
	\addcontentsline{toc}{subsection}{problem 8.9}
	\subsubsection*{(1)}
	对中位数$x_k$,设n为奇数,有$\frac{n+1}{2}-1$个元素小于$x_k$,$\frac{n+1}{2}-1$个元素大于$x_k$,则$\sum_{x_i>x_k}\frac{1}{n}=\sum_{x_i<x_k}\frac{1}{n}=\frac{1}{2}-\frac{1}{2n}<\frac{1}{2}.$对n为偶数,同样成立。
	\subsubsection*{(2)}
	建立以(w,x)为元素的结构体数组ori,w为权重,x为其对应下标。对其进行归并排序,时间复杂度为$O(n\log n)$。遍历ori,对权重进行累加,当权重和大于$\frac{1}{2}$时,对应结构体元素的x即为加权中位数,时间复杂度为$O(n)$。因此总的时间复杂度为$O(n\log n)$。\\
	$\begin{aligned}
	\hline
	&1.\textbf{Function}\ {KthNear1}\ (ori,n)\\
	&2.\hspace*{20pt}\text{对结构体数组ori使用归并排序}\\
	&3.\hspace*{20pt}cur\_weight:=0\verb|\\|\text{当前权重}\\
	&4.\hspace*{20pt}\textbf{for}\ i:= 1\ to\ n\ \textbf{do}\\
	&5.\hspace*{40pt}\textbf{if }cur\_weight+ori[i].w>\frac{1}{2} \textbf{ do }\\ 	&6.\hspace*{60pt}\textbf{return }ori[i].x\\
	&7.\hspace*{40pt}\textbf{else }\verb|\\|\text{更新权重和权重}\\
	&8.\hspace*{60pt}cur\_weight:=cur\_weight+ori[i].w\\
	\hline
	\end{aligned}
	$
	\subsubsection*{(3)}
	\paragraph{参考BFRPT算法}假设划分的权值和为tar,初始$tar=\frac{1}{2}$\\
	1. 将所有元素分成$\lceil\frac{n}{5}\rceil$组,每组5个元素\\
	2. 根据\hypertarget{problem 8.2}{problem 8.2}可知,比较6次找到5个元素中位数\\
	3. 递归使用BFRPT来找出各组中位数的中位数,记为m\\
	4. 基于m对元素进行划分,假设有x个元素小于m,n-x-1个元素大于m\\
	5. 计算x个元素的权值和T,若权值和T=tar,则m为加权平均数\\
	6. 若权值和T>tar,则在x个元素中递归寻找中位数,tar值不变\\
	7. 若权值和T<tar,则在n-x-1个元素中递归寻找中位数,tar=tar-T\\
	8. 时间复杂度$T(n)=T(n/5)+T(7n/10)+O(n)=\Theta(n)$,满足要求。
	\section*{CHAPTER 9}
	\markright{CHAPTER 9}
	\addcontentsline{toc}{section}{CHAPTER 9}
	\subsection*{problem 9.4}
	\markright{problem 9.4}
	\addcontentsline{toc}{subsection}{problem 9.4}
	\subsection*{problem 9.6}
	\markright{problem 9.6}
	\addcontentsline{toc}{subsection}{problem 9.6}
	\subsection*{problem 9.8}
	\markright{problem 9.8}
	\addcontentsline{toc}{subsection}{problem 9.8}
	\subsection*{problem 9.12}
	\markright{problem 9.12}
	\addcontentsline{toc}{subsection}{problem 9.12}
\end{document}