\documentclass[11pt]{ctexart}
\usepackage{tabularx}
\usepackage{array}
\usepackage{bm}
\usepackage{hyperref}
\usepackage{amsmath}
\usepackage{algorithm}
%\usepackage{algorithm2e}
\usepackage{algorithmicx}
\usepackage{algpseudocode}
\usepackage{fancyhdr}
\pagestyle{fancy}

\hypersetup{hypertex=true,
	colorlinks=true,
	linkcolor=blue,
	anchorcolor=blue,
	citecolor=blue}
\fancyhf{}
\chead{\textbf{算法设计与分析}}
\fancyhead[r]{\bfseries\thepage}
\fancyhead[l]{\bfseries\rightmark}
\renewcommand{\headrulewidth}{0.4pt} % 注意不用 \setlength
\renewcommand{\footrulewidth}{0pt}

\floatname{algorithm}{算法}
\renewcommand{\algorithmicrequire}{\textbf{输入:}}
\renewcommand{\algorithmicensure}{\textbf{输出:}}

\begin{titlepage}
	\title{\Huge\textbf{算法设计与分析 作业二}\\}
	\author{\Large\textbf{作者}:吴润泽 \and{\Large\textbf{学号}:181860109}\\
	\\
	\and {\Large\textbf{Email}:181860109@smail.nju.edu.cn}\\}
	\date{\Large\today}
\end{titlepage}

\begin{document}
	\maketitle
	\tableofcontents
	\newpage
	\section*{PART I}
	\markright{PART I}
	\addcontentsline{toc}{section}{PART I}
	\subsection*{problem 6.8}
	\markright{problem 6.8}
	\addcontentsline{toc}{subsection}{problem 6.8}
		
	\paragraph{算法分析}假定n总是k的倍数,且n和k都是2的幂。\\
	\hspace*{20pt}利用快排的思想,将数组从中间划分为两段$A[0\cdots n/2],\ A[n/2+1\cdots n]$,且左段元素小于右段元素。\\
	\hspace*{20pt}对于子序列继续递归划分,得到$A[0\cdots n/2^m],A[n/2^m+1\cdots n/2^{m-1}]\cdots A[n-n/2^m+1\cdots n]$,当$2^m=k\rightarrow m=\log k$时,划分完成。\\
	\hspace*{20pt}因此寻找中位数划分的函数时间复杂度应为$O(n)$,划分函数的递归方程为$W(n)=2W(n/2)+O(n)$,划分左右子段$\log k$次,方能使得总的时间复杂度达到$O(n\log k)$。
	\paragraph{具体算法实现请见算法k-sorted:\hyperref[k-sorted]{k-sorted算法}}
	\paragraph{算法时间复杂度}
	对于findk\_pos,每次递归代价为$O(n)$,每次子问题缩小为原来一半的规模,且子问题只有一个,可列出递归方程为$T(n)=T(n/2)+O(n)$,由主定理可以得出$T(n)=O(n)$。\\
	\hspace*{20pt}对于k\_sorted,每次递归代价为$O(n)$,每次子问题缩小为原来一半,而需要划分左右两序列,子问题为两个,可列出递归方程为$W(n)=2W(n/2)+O(n)$,注意结束条件为递归调用了$\log k$层,每层代价均为$O(n)$,因此时间复杂度为$O(n\log k)$满足题目要求。
	
	\begin{algorithm}
		\caption{k-sorted算法}
		\label{k-sorted}
		\begin{algorithmic}[1]
			\Require 待划分序列$A[1\cdots n]$,划分段数$k$
			\Ensure 划分后的的序列$A$
			\Function{findk\_pos}{$A,k\_pos,begin,end$} \verb|\\|返回该段数组第k小\\
			\verb|/*|
			利用快排思想,选定一个key,将大于key的元素放在其右边,小于key放于左边。\\
			判断key插入的位置是否为k,如果是则函数返回,如果插入位置大于k说明第k小位于左子序列对左边递归寻找,否则对右子序列递归寻找。
			\verb|*/|
			\State $split \gets begin,key \gets A[begin]$
			\For{$i\gets begin+1\ to\ end$}
				\State {$A[i]\leq key\ ?\ swap(A[++split],A[i])$}
			\EndFor
			\State{$split>k\_pos\ ?\ \Return\ findk\_pos(A,k\_pos,begin,split-1)$}
			\State{$split<k\_pos\ ?\ \Return\ findk\_pos(A,k\_pos,split+1,end)$}
			\State \Return{$split$}
			\EndFunction
			\Function {K\_SORTED}{$A,begin,end,k,count=1$}\\
			\verb|/*|
			count记录当前的段数,每次调用findk\_pos,A被分为[begin,mid]和[mid+1,end]两段,段数变为原来两倍,且左段元素小于右段,调用层数达到logk层算法结束,否则继续划分左右子序列
			\verb|*/|
			\State{$mid\gets (end-begin)/2+begin,count\gets count*2$}
			\State{$findk\_pos(A,mid,begin,end)$}
			\If{$count==k$}\\划分k段,算法结束
			\State \Return{$A$}
			\EndIf
			\State{$k\_sorted(A,begin,mid,k,count)$}
			\State{$k\_sorted(A,mid+1,end,k,count)$}
			\EndFunction
		\end{algorithmic}
	\end{algorithm}

	\newpage
	\subsection*{problem 6.9}
	\markright{problem 6.9}
	\addcontentsline{toc}{subsection}{problem 6.9}
	\paragraph{算法分析}同样利用快速排序的思想,令螺钉为A,螺母为B:\\
	\hspace*{20pt}1. 在A数组中拿一个,根据A和螺母的大小关系,可以分成三部分,B1:比螺钉小的,B2:比螺钉大的,B3:完全匹配的。\\
	\hspace*{20pt}2. 用B3,同样可以把A分为三部分,A1:比螺母小的,A2:比螺母大的,A3:完全匹配的。\\
	\hspace*{20pt}3. B1与A1匹配,B2与A2匹配,分别执行上述算法,直至全部匹配。
	\paragraph{具体算法实现请见算法match:\hyperref[match算法]{match算法}}
	\paragraph{算法时间复杂度}对于每次递归代价:\\
	\hspace*{20pt}1. 首先寻找分割点,遍历了一次数组,时间复杂度为$O(n)$。\\
	\hspace*{20pt}2. 之后根据分割点遍历A,B两数组将其分为两部分,时间复杂度为$O(n)$。\\
	\hspace*{20pt}3. 最后将分割后两序列进行递归操作继续划分。因此每次递归操作总的代价为$O(n)$。\\
	\hspace*{20pt}因此可推得算法的递推方程为$T(n)=2T(n/2)+O(n)$。根据主定理可得时间复杂度为$O(n\log n)$满足题目要求。
	\begin{algorithm}
		\caption{match算法}
		\label{match算法}
		\begin{algorithmic}[1]
			\Require 螺钉数组$A$,螺母数组$B$
			\Ensure 螺钉螺母对应下标一一匹配后的数组
			\Function{MATCH}{$A,B,l,r$}\\
			\verb|/*|找到分割点,mark记录B等于A首元素的下标,\\
			count记录B中小于A的个数\verb|*/|
			\State {$count \gets 0, mark\gets 0$}
			\For{$i\gets l \ to\ r$}
			\State{$A[l]==B[i]\ ?\ mark=i$}
			\State{$A[l]>B[i]\ ?\ count+=1$}
			\EndFor\\
			\verb|/*|为B和A的左半部分分配count个元素\verb|*/|
			\State{$swap(A[l],A[l+count]),\ swap(B[mark],B[l+count])$} 
			\State{$mark\gets mark+count, i\gets l,\ j \gets r$}
			\While{$i<mark$\ \textbf{and}\ $j>mark$}\verb|\\|将a分成两部分
			\While{$i<mark$\ \textbf{and}\ $a[i]<b[mark]$}\State{$i\gets i+1$}
			\EndWhile
			\While{$j>mark$\ \textbf{and}\ $a[j]<b[mark]$}\State{$j\gets j-1$}
			\EndWhile
			\State{$swap(a[i++],a[j--])$}
			\EndWhile
			
			\State{$i\gets l,\ j \gets r$}
			\While{$i<mark$\ \textbf{and}\ $j>mark$}\verb|\\|将b分成两部分
			\While{$i<mark$\ \textbf{and}\ $b[i]<a[mark]$}\State{$i\gets i+1$}
			\EndWhile
			\While{$j>mark$\ \textbf{and}\ $b[j]<a[mark]$}\State{$j\gets j-1$}
			\EndWhile
			\State{$swap(b[i++],b[j--])$}
			\EndWhile
			\State{$l<mark\ ? \ match(A,B,l,mark-1)$}
			\State{$r>mark\ ? \ match(A,B,mark+1,r)$}
			\EndFunction
		\end{algorithmic}
	\end{algorithm}
	\newpage
	\subsection*{problem 6.10}
	\markright{problem 6.10}
	\addcontentsline{toc}{subsection}{problem 6.10}
	\subsubsection*{(1)}
	\paragraph{算法分析}利用归并排序的思想,在归并排序中合并左右数组A,B\\
	\hspace*{20pt}1. 由归并排序定义可知,A、B两数组已经有序。\\
	\hspace*{20pt}2. 在合并过程中,就需要计算a[i],b[j]分别来自左右两部分的逆序对数,同时遍历两个数组,对于遍历的两个数进行比较大小,如果$a[i]<b[0]$显然没有逆序。\\
	\hspace*{20pt}4. 如果$a[i]>b[j]$,那么$a[i+1\cdots n]>b[j]$均成立,即逆序对数有$n-i+1$个。
	因此遍历时如遇到$a[i]>b[j]$,逆序对数加$n-i+1$即可。
	\paragraph{具体算法实现请见算法MergeSort:\hyperref[MergeSort算法]{MergeSort算法}}
	\paragraph{算法时间复杂度}与归并排序算法相同,在合并函数中仅添加一条赋值语句,复杂度仍为$O(n\log n)$符合题目要求。
	\subsubsection*{(2)}
	\paragraph{算法分析}
	利用归并排序的思想,在归并排序中合并左右数组A,B。
	若$a[i]>b[j]\&a[i]>C\cdot b[j]$,则$a[i+1\cdots n]>C\cdot b[j]$均成立,与 \hyperref[MergeSort算法]{MergeSort算法}\ 相似。当$a[i]>C\cdot b[j]$时,遍历a数组,找到$a[tem]>C\cdot b[j]$,更新总的逆序对个数即可。
	\paragraph{具体算法实现请见算法GeneralPair:\hyperref[GeneralPair算法]{GeneralPair算法}}
	\paragraph{算法时间复杂度}
	在$a[i]>C\cdot b[j]$时,遍历左数组,每次合并的时间复杂度为$O(n^2)$。总的递推方程为$T(n)=2T(\frac{n}{2})+O(n^2)$,时间复杂度变为$O(n^2)$。

	\begin{algorithm}
		\caption{MergeSort算法}
		\label{MergeSort算法}
		\begin{algorithmic}[1]
			\Require 无序序列$A,l,r$
			\Ensure  总逆序对数和$sum$
			\State{$sum\gets 0$}
			\Function {MERGESORT}{$A,l,r$}
			\State{$l==r\ ?\ \Return{}$}
			\State{$mid\gets \frac{l+r}{2}$}
			\State{$MergeSort(A,l,mid),\ MergeSort(A,mid+1,r)$}
			\State{$Merge(A,l,mid,r)$}
			\EndFunction
			\Function{Merge}{$A,l,mid,r$}
			\State{$n1\gets mid-l+1,\ n2\gets r-mid$}\\
			\verb|Let L[1..(n1+1)] and R[1..(n2+1)] be new arrays|
			\For{$i\gets1\ to\ n1$}
			\State{$L[i]\gets A[l-i+1]$}
			\EndFor
			\For{$i\gets1\ to\ n2$}
			\State{$R[i]\gets A[mid+i]$}
			\EndFor
			\State{$L[n1+1]\gets \infty,\ R[n2+1]\gets \infty$}
			\State{$i\gets 1,\ j\gets 1$}
			\For{$k\gets l\ to\ r$}
				\If{$L[i]<R[j]$}
				\State{$A[k]\gets L[i++]$}
				\Else
				\State{$A[k]\gets R[j++]$}
				\State{$sum\gets sum+n1-i+1$}\verb|\\|仅添加这句,更新sum逆序对和
				\EndIf
			\EndFor
			\EndFunction
		\end{algorithmic}
	\end{algorithm}
	\begin{algorithm}
		\caption{GeneralPair算法}
		\label{GeneralPair算法}
		\begin{algorithmic}[1]
			\Require 无序序列$A,l,r,C$
			\Ensure  总广义逆序对数和$sum$
			\State{$sum\gets 0$}
			\Function {MERGESORT}{$A,l,r,C$}
			\State{$l==r\ ?\ \Return{}$}
			\State{$mid\gets \frac{l+r}{2}$}
			\State{$MergeSort(A,l,mid,C),\ MergeSort(A,mid+1,r,C)$}
			\State{$Merge(A,l,mid,r)$}
			\EndFunction
			\Function{Merge}{$A,l,mid,r$}
			\State{$n1\gets mid-l+1,\ n2\gets r-mid$}\\
			\verb|Let L[1..(n1+1)] and R[1..(n2+1)] be new arrays|
			\For{$i\gets1\ to\ n1$}
			\State{$L[i]\gets A[l-i+1]$}
			\EndFor
			\For{$i\gets1\ to\ n2$}
			\State{$R[i]\gets A[mid+i]$}
			\EndFor
			\State{$L[n1+1]\gets \infty,\ R[n2+1]\gets \infty$}
			\State{$i\gets 1,\ j\gets 1$}
			\For{$k\gets l\ to\ r$}
			\If{$L[i]<R[j]$}
			\State{$A[k]\gets L[i++]$}
			\Else
			\For{$tem\gets i\ to\ n1$}\verb|\\|找到广义逆序对
			\If{$L[i]>C\cdot R[j]$}
			\State{$sum\gets sum+n1-i+1$}\verb|\\|更新sum逆序对和
			\EndIf
			\EndFor
			\State{$A[k]\gets R[j++]$}
			\EndIf
			\EndFor
			\EndFunction
		\end{algorithmic}
	\end{algorithm}
	\newpage
	\subsection*{problem 6.13}
	\markright{problem 6.13}
	\addcontentsline{toc}{subsection}{problem 6.13}
	\subsubsection*{(1)}
	\paragraph{\textcircled{1}证明}易知当R中元素个数为13倍数相同元素均为k个,则R中存在13个常见元素。假设可以有14个常见元素:则这14个常见元素个数总和$sum>=14\cdot\lceil\frac{n}{13}\rceil>n$,与总元素个数为n矛盾。\\
	因此R中存在最多13个不同的常见元素,证毕。\\
	\textbf{\textcircled{2}证明}x为$R[1..n]$中常见元素,则设x在$R[1..\lfloor\frac{n}{2}\rfloor]$中有$k_1$个,在$R[\lfloor\frac{n}{2}\rfloor+1..n]$中有$k_2$个,则$k_1+k_2>=\lceil\frac{n}{13}\rceil$,假设x在两个数组中均不为常见元素,则$k_1<\lceil\frac{n}{26}\rceil,\ k_2<\lceil\frac{n}{26}\rceil$,即$k_1+k_2<2\cdot\lceil\frac{n}{26}\rceil<\lceil\frac{n}{13}\rceil$,产生矛盾,因此x至少是两个数组中一个数组的常见元素。\\
	\textbf{\textcircled{3}算法设计}\\
	1. 当数组元素小于k时,返回数组元素,否则递归划分为两部分。\\
	2. 得到的两部分结果依次与原数组进行统计,如果元素出现次数不低于$\lceil\frac{n}{k}\rceil$,则加入结果集合,重复前两步直至算法完成。\\
	3. \textbf{时间复杂度}合并时两部分结果个数不超过2k,遍历原始数组统计个数,即合并复杂度为$O(n)$,
	递归产生两个子问题,子问题规模为原来一半,可列出递归方程为$T(n)=2T(\frac{n}{2})+O(n)$,根据主定理可得$T(n)=O(n\log n)$。\\
	$
	\label{Majority算法}
	\begin{aligned}
	\hline
	&\textbf{算法5 }Majority\text{ 算法}\\
	\hline
	&1.\textbf{Function}\ Majority\ (Ori,A,n,L,R,k)\verb|          \\|\text{Ori为原始数组拷贝}\\
	&2.\hspace*{20pt}\textbf{if}\ len(A)\leq k\ \textbf{do }\textbf{return }A
	\verb|               \\|\text{递归终点,返回数组即可}\\
	&4.\hspace*{20pt}mid\gets (L+R)/2\\
	&5.\hspace*{20pt}LS \gets Majority(Ori,A,L,mid,k)\verb|          \\|\text{进行递归划分}\\
	&6.\hspace*{20pt}RS \gets Majority(Ori,A,mid+1,R,k))\\
	&7.\hspace*{20pt}ans\gets set\_init() \verb|                    \\|\text{初始化集合ans}\\
	&8.\hspace*{20pt}\textbf{for each }\ item\textbf{ in } LS\textbf{ and }RS\textbf{ do}\\
	&9.\hspace*{40pt}\textbf{if }count(item,Ori)\geq\lceil\frac{n}{k}\rceil\textbf{ do}\verb|    \\|\text{count函数用于统计元素在原数组中出现次数}\\
	&10.\hspace*{55pt}ans.add(item)\verb|            \\|\text{将满足结果的item加入集合ans}\\
	&11.\hspace*{15pt}\textbf{return }ans\\
	\hline
	\end{aligned}
	$
	\subsubsection*{(2)}
	正常工作
	\subsubsection*{(3)}
	\paragraph{存在}采用摩尔投票算法\\
	1. 假设数组个数为奇数,其中次数$\geq(n+1)/2$的数最多1个,如果个数为偶数,则选取最后一个元素,判断是否为众数,这样数组个数再次变为奇数。\\
	2. 预设众数为第一个,遍历数组,当众数和当前数相同,频率加一。\\
	3. 如果不相同,则频率减一(相当于两元素抵消),如果频率变为0,则当前元素作为众数,直到遍历所有元素。\\
	4. 再进行一次循环判断结果是否出现超过n/2次即可,时间复杂度为$O(n)$。
	$
	\label{MooreMajority算法}
	\begin{aligned}
	\hline
	&\textbf{算法6 }MooreMajority\text{ 算法}\\
	\hline
	&1.\textbf{Function}\ MooreMajority\ (A[1\cdots n])\\
	&2.\hspace*{20pt}\textbf{if }n\%2==0\textbf{ do}\\
	&3.\hspace*{40pt}\textbf{if }count(A[n])\geq\frac{n}{2}\textbf{ return }A[n]\\
	&4.\hspace*{40pt}\textbf{else }n\gets n-1\\
	&5.\hspace*{20pt}res\gets A[1],\ count\gets1\\
	&6.\hspace*{20pt}\textbf{for}\ i\gets 2\ to\ n\ \textbf{do}\\
	&7.\hspace*{40pt}\textbf{if}\ res==A[i]\ \textbf{then}\ count\gets count+1\\
	&8.\hspace*{40pt}\textbf{else if}\ count==0\ \textbf{then}\ res\gets A[i], count\gets1\\
	&9.\hspace*{40pt}\textbf{else}\ count\gets count-1\\
	&10.\hspace*{15pt}\textbf{for}\ i\gets 1\ to\ n\ \textbf{do}\\
	&11.\hspace*{35pt}\textbf{if}\ res==A[i]\ \textbf{then}\ count\gets count+1\\
	&12.\hspace*{15pt}\textbf{return}\ count\geq\lceil\frac{n}{2}\rceil\ ?\ res\ :\ 0\\
	\hline
	\end{aligned}
	$
	\subsubsection*{(4)}
	由(3)可知采用类似的摩尔投票算法,\textbf{常见元素问题的下界为}$\Omega(n)$。而比较排序的下界为$\Omega(n\log n)$,因此比比较排序更容易。
	\subsection*{problem 6.15}
	\markright{problem 6.15}
	\addcontentsline{toc}{subsection}{problem 6.15}
	\subsubsection*{(1)}
	\paragraph{\textcircled{1}排序算法}
	1.对输入的n个点进行排序,排序规则为按x由小到大排序(x相等时,按y由小到大排序)。\\
	\hspace*{20pt}2. x最大且y最大的点(即排序后的最后一个点)一定为$maxima$,且前面点的x坐标小于等于后面的x坐标。\\
	\hspace*{20pt}3. 从第N个点向前遍历每个点,$maxy$记录当前极大点中y的最大值,当第i个点的大于maxy并且第i-1个点(如果存在)的横坐标不等于第i个点则该点即为极大点,直至遍历所有点。\\
	\hspace*{20pt}4. \textbf{时间复杂度:}排序采取归并排序时间复杂度为$O(n \log n)$,遍历排序后数组时间复杂度为$O(n)$,因此时间复杂度为$O(n\log n)$。\\
	$
	\label{MaximaSort算法}
	\begin{aligned}
	\hline
	&\textbf{算法7 }MaximaSort\text{ 算法}\\
	\hline
	&\textbf{输入:}\text{结构体数组A   }\ \ \ \textbf{输出:}\text{maxima点集ans}\\
	&1.\textbf{Function}\ MaximaSort\ (A[1\cdots n])\\
	&2.\hspace*{20pt}\text{使用归并排序结构数组A进行排序}\\
	&3.\hspace*{20pt}maxy\gets1,\ A[0].x\gets -\infty\verb| \\|\text{A[0]作为哨兵}\\
	&4.\hspace*{20pt}\textbf{for}\ i\gets n\ to\ 1\ \textbf{do}\\
	&5.\hspace*{40pt}\textbf{if}\ A[i].y>maxy\ \textbf{and}\ A[i-1].x<A[i].x\ \textbf{then}\\ 
	&6.\hspace*{60pt}maxy\gets y\\
	&7.\hspace*{60pt}ans.add(A[i])\\
	&8.\hspace*{20pt}\textbf{return}\ ans\\
	\hline
	\end{aligned}
	$
	\paragraph{\textcircled{2}分治算法}
	1. 当数组A中仅有一个点时返回该点。\\
	2. 取数组A中所有点的x坐标的中值点,将所有点划分为两部分S1,S2。\\
	3. 递归方式继续划分A1和A2,得到maxima集合R1和R2。\\
	4. 显然R2中maxima即为当前A的maxima,仅需要判断R1中的maxima是否符合即可。\\
	5. 取R2中的y坐标最大值$y_{max}$,遍历R1中所有坐标,若点的y坐标大于$y_{max}$,则加入maxima集合。\\
	6. 回到第1步,直至算法完成。\\
	7. \textbf{时间复杂度:}合并的比较代价为$O(n)$,递归产生两个子问题,子问题规模为原来一半,可列出递归方程为$T(n)=2T(\frac{n}{2})+O(n)$,根据主定理可得$T(n)=O(n\log n)$。\\
	$
	\label{Maxima算法}
	\begin{aligned}
	\hline
	&\textbf{算法8 }Maxima\text{ 算法}\\
	\hline
	&\textbf{输入:}\text{结构体数组A   }\ \ \ \textbf{输出:}\text{maxima点集ans}\\
	&1.\textbf{Function}\ Maxima\ (A[1\cdots n])\\
	&2.\hspace*{20pt}\textbf{find }x_{mid} \textbf{ in}\ A\\
	&3.\hspace*{20pt}\textbf{divide }A\textbf{ into }A1,A2\textbf{ based on }x_{mid}\\ 
	&4.\hspace*{20pt}R1\gets Maxima(A1)\textbf{ and }R2\gets Maxima(A2)\\
	&5.\hspace*{20pt}ans.add(R2)\textbf{ and find }the\ y_{max} \textbf{ in }R2\\
	&8.\hspace*{20pt}\textbf{for each } item \textbf{ in }R1\textbf{ do}\\
	&9.\hspace*{40pt}\textbf{if }item.y>y_{max}\textbf{ then }ans.add(item)\\
	&10.\hspace*{15pt}\textbf{return}\ ans\\
	\hline
	\end{aligned}
	$
	
	\subsubsection*{(2)}
	$
	\label{WrongMaxima算法}
	\begin{aligned}
	\hline
	&\textbf{算法9 }WrongMaxima\text{ 算法}\\
	\hline
	&1.\textbf{Function}\ WrongMaxima\ (A[1\cdots n])\\
	&2.\hspace*{20pt}\textbf{if }n\leq1\textbf{ then return }A\\
	&3.\hspace*{20pt}\textbf{find }{mid\_{point}} \textbf{ in}\ A\\
	&4.\hspace*{20pt}\textbf{divide }A\textbf{ into }A1,A2,A3,A4\textbf{ based on }{mid\_{point}}\\ 
	&5.\hspace*{20pt}R2\gets Maxima(A2)\textbf{ and }R3\gets Maxima(A3)\textbf{ and }R4\gets Maxima(A4)\\
	&6.\hspace*{20pt}ans.add(R4)\textbf{, find }\ x_{max},\ y_{max} \textbf{ in }R4\\
	&6.\hspace*{20pt}\textbf{for each } item \textbf{ in }R2\textbf{ do}\verb|\\|\text{左上大于右上的最大纵坐标则满足}\\
	&7.\hspace*{40pt}\textbf{if }item.y>y_{max}\textbf{ then }ans.add(item)\\
	&6.\hspace*{20pt}\textbf{for each } item \textbf{ in }R3\textbf{ do}\verb|\\|\text{右下大于右上的最大值横坐标则满足}\\
	&7.\hspace*{40pt}\textbf{if }item.x>x_{max}\textbf{ then }ans.add(item)\\
	&8.\hspace*{20pt}\textbf{return}\ ans\\
	\hline
	\end{aligned}
	$\\
	\textbf{不正确:}原问题与划分后的子问题可能均没有左下象限的点,则子问题的规模没有变为原来的$3/4$,递归方程退化为$T(n)=3T(\frac{n}{3})+O(n)=O(n\log n)$。
	\subsubsection*{(3)}
	算法核心也是一种基于比较的算法,下界应与基于比较的排序算法下界相同,均为$\Omega(n\log n)$,具体证明方法由于自身能力有限,故不能给出。
	\newpage
	\section*{PART II}
	\markright{PART II}
	\addcontentsline{toc}{section}{PART II}
	\subsection*{problem 7.1}
	\markright{problem 7.1}
	\addcontentsline{toc}{subsection}{problem 7.1}
	\subsubsection*{证明}
	$
	\begin{aligned}
	\hline
	&\text{假设}2^k\leq h\leq2^{k+1}-1,\ k\in N\\
	&2^{k-1}+1\leq\lfloor\frac{h}{2}\rfloor+1\leq2^k\\
	&\lceil\log(2^{k-1}+1)\rceil=k,\ \lceil\log(2^k)\rceil=k\\
	&\Rightarrow \lceil\log(\lfloor\frac{h}{2}\rfloor+1)\rceil=k\hspace{40pt}(1)\\
	&2^k+1\leq h+1\leq2^{k+1}\rightarrow\log(2^k+1)>k,\ \log(2^{k+1})=k+1\\
	&\Rightarrow \lceil\log(h+1)\rceil=k+1\hspace{35pt}(2)\\
	&\text{因此}\lceil\log(\lfloor\frac{h}{2}\rfloor+1)\rceil+1=\lceil\log(h+1)\rceil=k+1\text{证毕。}\\
	\hline
	\end{aligned}
	$\\
	\paragraph{堆的层数计算}
	h可以理解为堆的节点个数,设堆高度为k\\
	则根据堆的完全二叉树性质可知$2^k\leq h\leq2^{k+1}-1$,则$\lceil\log(h+1)\rceil$即为计算堆层数即k+1层。\\
	由于$2^{k-2}-1<\lfloor\frac{h}{2}\rfloor<2^{k}-1$,因此具有$\lfloor\frac{h}{2}\rfloor$结点个数的堆高度为k-1,即
	$\lceil\log(\lfloor\frac{h}{2}\rfloor+1)\rceil$即堆层数为k层,加上最后一层则为k+1层,因此两式均为计算堆的层数。
	\newpage
	\subsection*{problem 7.2}
	\markright{problem 7.2}
	\addcontentsline{toc}{subsection}{problem 7.2}
	\subsubsection*{算法设计}
	1. 给定最大堆为heap1,新建最大堆heap2,存储结构元素(value,index),比较原则为value。\\
	\hspace*{20pt}2. 从heap1中取出根节点将其放入heap2。\\
	\hspace*{20pt}3. 从heap2中弹出根节点Node,K减一,当K减为0,返回Node的value即可。\\
	\hspace*{20pt}4. 插入根节点对应的左右子节点到heap2中。\\
	\hspace*{20pt}5. 回到第2步。\\

	\paragraph{算法分析}每次插入元素均为heap1当前最大元素,heap2 pop第i次即为第i大元素。\\
	$
	\label{Kth-Maxheap算法}
	\begin{aligned}
	\hline
	&\textbf{算法11 }KthMaxheap\text{ 算法}\\
	\hline
	&\textbf{输入:}\text{最大堆heap1[1...n],K   }\ \ \ \textbf{输出:}\text{第K大元素}\\
	&1.\textbf{Function}\ KthMaxheap\ (heap1,K)\\
	&2.\hspace*{20pt}heap2\gets init(heap1[0],0)\\
	&3.\hspace*{20pt}\textbf{for}\ i\gets 1\ to\ K\ \textbf{do}\\
	&4.\hspace*{40pt}(value,index)\gets heap2.pop()\\
	&5.\hspace*{40pt}heap2.insert(heap1[index*2+1],index*2+1)\\
	&6.\hspace*{40pt}heap2.insert(heap1[index*2],index*2)\\
	&7.\hspace*{20pt}\textbf{return}\ value\\
	\hline
	\end{aligned}
	$
	\paragraph{时间复杂度}heap2的元素个数最大为k+1,因此每次插入删除操作复杂度均为$O(\log(k))$。\\
	最多2k次插入和k次删除,因此总的时间复杂度为$O(k\log k)$,符合要求。
	\subsection*{problem 7.3}
	\markright{problem 7.3}
	\addcontentsline{toc}{subsection}{problem 7.3}
	对于d叉堆中第m层的第n个元素(m,n均从1开始),则它在数组中的下标为:
	$$I(m,n)=(1+d+d^2+\cdots+d^{m-2})+n=\frac{d^{m-1}-1}{d-1}+n$$
	对于其父节点,则为第m-1层的第$\lfloor\frac{n}{d}\rfloor$个元素,因此下标为:\\
	$
	\begin{aligned}
	PARENT(m,n)&=I(m-1,\lceil\frac{n}{d}\rceil)\\
	&=\frac{d^{m-2}-1}{d-1}+\lfloor\frac{n}{d}\rfloor\\
	&=\frac{d^{m-1}-d}{d(d-1)}+\lfloor\frac{n+d-1}{d}\rfloor\\
	&=\lfloor\frac{d^{m-1}-d}{d(d-1)}+\frac{n+d-1}{d}\rfloor\\
	&=\lfloor\frac{1}{d}\frac{d^{m-1}-d}{d-1}+n+d-1\rfloor\\
	&=\lfloor\frac{1}{d}\frac{d^{m-1}-1}{d-1}+n+d-2\rfloor\\
	&=\lfloor\frac{I(m,n)+d-2}{d}\rfloor\\
	\end{aligned}
	$\\
	对于其子节点,则为第m+1层的第d(n-1)+j个元素,因此下标为:\\
	$
	\begin{aligned}
	CHILD(m,n,j)&=(1+d+d^2+\cdots+d^{m-1})+d(n-1)+j\\
	&=\frac{d^m-1}{d-1}+d(n-1)+j\\
	&=\frac{d^{m}-d}{d-1}+d(n-1)+1+j\\
	&=d(\frac{d^{m-1}-1}{d-1}+n-1)+j+1\\
	&=d(I(m,n)-1)+j+1\\
	\end{aligned}
	$\\
	因此当元素数组下标为i的父节点和第j个字节点下标为,\\
	$D-ARY-PARENT(i)=\lfloor\frac{i-2}{d}+1\rfloor$\\
	$D-ARY-CHILD(i,j)=d(i-1)+j+1$,得证。
	\subsection*{problem 7.4}
	\markright{problem 7.4}
	\addcontentsline{toc}{subsection}{problem 7.4}
	对于完美二叉树,设其树的高度为h,则最底层共有$2^h$个树叶。\\
	其节点总数为$2^{h+1}-1$,其所有节点的高度和为
	$2^{h+1}-h-2.$  \textbf{(1)}\\
	\hspace*{20pt}对堆节点个数$n$进行归纳,易得$n=1时,0\leq1-1=0$,满足。\\
		\hspace*{20pt}假设$n\leq k$,令堆的总高度和与节点总数差最大值为$gap(k)$,均有$gap(n)\leq-1.$\\
		\hspace*{20pt}当$n=k+1$时,根据堆的性质,堆的左右子树至少有一个为完美二叉树。\\
		\hspace*{20pt}1.若堆为完美二叉树,设堆高度为h,高度和与节点总数差为:\\
		$$
		gap(k+1)=2^{h+1}-h-2-(2^{h+1}-1)=-h-1\leq-1.
		$$
		\hspace*{20pt}2.当左子树为完美二叉树,其高度为$h-1$,其高度和为$2^{h}-h-1$,节点总数为$2^{h}-1$,对右子树由归纳假设,其高度和与节点总数差最大值为$-1$.\\
	则:$$\begin{aligned}
	gap(k+1)&\leq h-1+(2^{h}-h-1)-(2^{h}-1)+max(gap(k_i))=-2\leq -1\\
	\end{aligned}
	$$
	\hspace*{20pt}3.当左子树不为完美二叉树而右子树为完美二叉树,高度为$h-2$,其高度和为$2^{h-1}-h$,节点总数为$2^{h-1}-1$,对左子树由归纳假设,其高度和与节点总数差最大值为$-1$.\\
	则:
	$$\begin{aligned}
	gap(k+1)&\leq h-1+(2^{h-1}-h)-(2^{h-1}-1)+max(gap(k_i))=-1\\
	\end{aligned}
	$$
	\hspace*{20pt}当最底层仅有一个节点时,节点个数和所有节点高度和的差值为
	$$\begin{aligned}
	gap(n)=h-1+((2^{h-1}-1)-(2^{h-1}))+((2^{h-1}-h)-(2^{h-1}-1))=-1
	\end{aligned}
	$$
	\hspace*{20pt}即n个节点的堆中,所有节点的高度之和最多为n-1,在最底层仅有一个节点时等号成立。

	\subsection*{problem 7.6}	
	\markright{problem 7.6}
	\addcontentsline{toc}{subsection}{problem 7.6}
	\subsubsection*{算法设计}
	1. 使用归并排序将每个单词内部字母按照从小到大排序,使得查找易位词时方便操作。\\
	\hspace*{20pt}2. 建立结构体$Data$,其中包含原始字符串ori,排序后字符串key,以及该字符串出现的次数count。\\
	\hspace*{20pt}3. 依次遍历每个单词,构造结构$data\_cur$,以判断是否在集合$data\_set$中:\\
	\hspace*{40pt}\textcircled{1}. 如果在集合中,结构体为$data\_old$,根据题意换为字典序小的字符串,并更新出现次数,即$count+1$。\\
	\hspace*{40pt}\textcircled{2}. 如果没有在集合中,则将该结构体加入$data\_set$即可。\\
	\hspace*{20pt}4. 遍历整棵树,将$count>0$的节点的ori输出。算法完成。\\
	\paragraph{具体算法实现请见算法FindTrans:\hyperref[FindTrans算法]{算法FindTrans}}
	\paragraph{时间复杂度}对每个单词内部进行归并排序,设单词总数为$num$,单词长度为$len(x_i)$时间复杂度为$\sum_{i=0}^{num}len(x_i)\log(len(x_i))=O(n\log n)$;集合每次的插入操作时间复杂度为$\log n$,进行次数为$O(n)$,因此维护集合有序性的时间复杂度为$O(n\log n)$。因此总的时间复杂度为$O(n\log n)$。\\
	\begin{algorithm}
		\setcounter{algorithm}{11}
		\caption{FindTransposition算法}
		\label{FindTrans算法}	
		\begin{algorithmic}[1]
		\Require 单词序列oris,单词个数num
		\Ensure  易位词序列res
		\Function{FindTrans}{keys,num}
		\State{$data\_set\gets set\_init(),\ res\gets set\_init()$}
		\For{$i\gets 0\ to\ num$}
		\State{$key\gets merge\_sort(oris[i])\verb|\\|\text{得到归并排序后的关键词key}$}
		\State{$data\_cur(key,oris[i],0)\verb|\\|\text{构造结构体data\_cur}$}
		\If{$find(data\_set,data\_old)==True$}
		\State{$\verb|/*|\text{如果找到相同的key,则返回对应的结构体}data\_old\verb|*/|$}
		\State{$data\_old.count\small{++}$}
		\State{$data\_old.ori\gets min(data\_old.ori,oris[i])$}	
		\State{$\verb|/*|\text{更新count计数,并将ori设为字典序较小的字符串}\verb|*/|$}
		\Else
		\State{$data\_set.add(data\_cur)\verb|\\|\text{如果没有则将data\_cur加入集合中}$}	
		\EndIf
		\EndFor
		\ForAll{$item \textbf{ in }data\_set$}
		\If{$item.count\geq1$}
		\State{$res.add(item.ori)\verb|\\|\text{如果出现次数不为0,即为易位词,加入res}$}
		\EndIf
		\EndFor
		\State \Return{$res$}
		\EndFunction
		\end{algorithmic}
	\end{algorithm}
\end{document}